\documentclass{article}
\usepackage[T2A]{fontenc}
\usepackage[utf8]{inputenc}
\usepackage{indentfirst}
\usepackage{rotating}

\usepackage[colorlinks=true, allcolors=blue]{hyperref}




\title{Open and Subjective Curriculum \\ 04}
\author{Konczer J.}
\date{November 2021}

\begin{document}

\maketitle

\section*{\nameref{sec:toc}}

\section{Introduction}

This document is a summary based on my experience as a Mentor at the \href{https://milestone-institute.org/}{Milestone Institute}, and as a former high school student at a Hungarian school in Slovakia, and University student at \href{https://www.bme.hu/?language=en}{BUTE} and \href{https://www.elte.hu/en/}{ELTE} in Budapest.

Hopefully, this document will be helpful not only for those, who have similar interests to me, but it can bring us one step further towards a society, where more experienced people share their knowledge with younger ones, and where quality education is a shared value and a social enterprise.


\section{General}

Online courses on high school level in English
\begin{itemize}
\item \href{https://thecrashcourse.com/}{Crash Course}
\item \href{https://www.khanacademy.org/}{Khan Academy}
\item \href{https://www.bbc.co.uk/bitesize/levels/z98jmp3}{BBC bitesize} (brief summaries of general subjects/concepts)
\end{itemize}
Online materials in Hungarian
\begin{itemize}
    \item \href{https://zanza.tv/}{zanza.tv} (not all subjects)
    \item \href{https://www.youtube.com/user/videotanar/}{Videotanár} private tutoring style high school material
    \item \href{https://www.youtube.com/playlist?list=PLEiwVd6n_Q_38BxQQDqYZE6bdMQAZxbX2}{M5-Érettségi} (High school final exams (érettségi) preparation)
\end{itemize}

A general University level introductory series is the \href{https://www.veryshortintroductions.com/browse}{Very Short Introductions} (not free to access)

Online University level course
\begin{itemize}
    \item \href{https://www.coursera.org/}{Coursera}
    \item \href{https://www.edx.org/}{EdX}
    \item \href{https://ocw.mit.edu/index.htm}{MIT OpenCourseWare}
\end{itemize}

Hungarian University level lectures
\begin{itemize}
    \item \href{https://bme.videotorium.hu/}{BME}
    \item \href{http://videotorium.hu/}{Academic lectures} (in a non structured form)
\end{itemize}


\section{Using the Internet}

The internet is an inevitable source of knowledge and information, so it is essential to know how to use it effectively.

First of all, I would highlight a short course: \href{https://www.youtube.com/playlist?list=PL8dPuuaLjXtN07XYqqWSKpPrtNDiCHTzU}{How to navigate digital media} on Crash Course. It talks about the essential features of Online Media.


\href{https://www.wikipedia.org/}{Wikipedia} is a surprisingly good starting point to gain information. (A nice way to listen to the ongoing creation of Wikipedia can be found \href{http://listen.hatnote.com/#nowelcomes,en}{here}.)

Although, Wikipedia is a good starting point, and can give an overall picture, it is important to be able to follow the references, which are provided at the end of the articles.
These are usually books and/or scientific articles. \footnote{Other encyclopedia type sites, which I would mention are: \href{https://www.britannica.com/}{Encyclopædia Britannica}, \href{http://www.scholarpedia.org/article/Main_Page}{Scholarpedia}, \href{https://www.encyclopedia.com/}{encyclpedia.com} and for philosophy \href{https://plato.stanford.edu/index.html}{Stanford Encyclopedia of Philosophy}.}

It is usually possible to access these materials, however, one has to take ethical and practical considerations into account as well.
These materials are accessible mainly by violating copyrights and/or paywalls. This is not only legally questionable, but one has to think about the sustainability of quality knowledge and content making. Copyright can be \href{https://link.springer.com/chapter/10.1007/978-981-10-3984-3_7}{debated} (see also \href{https://www.youtube.com/watch?v=rFMl0stqai0}{here}, \href{https://en.wikipedia.org/wiki/Criticism_of_copyright}{here} and \href{https://en.wikipedia.org/wiki/Right_to_science_and_culture}{here} (\href{https://www.ted.com/talks/lawrence_lessig_laws_that_choke_creativity}{here})), but it is part of the Human Rights (\href{https://www.humanrights.com/what-are-human-rights/videos/copyright.html}{Article 27}), and is implemented in USA, EU and various national legal systems. Copyright and Intellectual Property rights play an important role in financing content creators, which means, that if one accesses quality contents without paying a contribution, then the financial support of the creator should be made in some another way.
\href{https://www.patreon.com/}{Patreon} is a possible way to support content creators, which can produce free and quality content, however the distribution of financial support for authors is still a challenging problem (see \href{https://ko-fi.com/}{buy a coffee} as a recent attempt), which still needs creative problem solvers in the future.

After these legal, ethical and financial comments, I would like to mention two sites to access scientific books and articles: \href{https://en.wikipedia.org/wiki/Sci-Hub}{Sci-Hub} for articles under paywall and \href{https://en.wikipedia.org/wiki/Library_Genesis}{Library Genesis} for books.

On the internet credibility of a source is key, which should be determined. In case of scientific articles the Journal where they appeared is an important indication of their credibility. In general, \href{https://en.wikipedia.org/wiki/Peer_review}{peer-reviewed} articles are considered more reliable than non peer-reviewed ones. Some most popular scientific journals are: \href{https://www.nature.com/}{Nature}, \href{https://www.sciencemag.org/}{Science}, \href{https://journals.plos.org/plosone/}{Plos One}, \href{https://journals.aps.org/}{Phys. Rew.} (There are some sites, where quality non peer-reviewed content can be found. The main example is \href{https://arxiv.org/}{arXiv}, other mentionable \href{https://en.wikipedia.org/wiki/List_of_preprint_repositories}{preprint servers} are \href{https://hal.archives-ouvertes.fr/}{HAL} and \href{https://www.biorxiv.org/}{bioRxiv}.)

Another obvious starting point to gain information from is \href{https://www.google.com/}{Google}, however, there are many \href{https://ahrefs.com/blog/google-advanced-search-operators/}{operators} which can be used to search more effectively.
An alternative to Google is \href{https://duckduckgo.com/}{DuckDuckGo}, which lists different sites, and gives more privacy than Google.\footnote{Google's Russian and Chinese alternatives are \href{https://yandex.com/}{Yandex} and \href{https://www.baidu.com/}{Baidu} respectively.}

To have a feeling about the structure of the ``internet'', one can browse an outdated, but still more or less relevant \href{https://internet-map.net/}{map} of the internet. A list of the biggest websites (globally and by country) can be found on \href{https://www.alexa.com/topsites}{Alexa}. (Or a more recent poster can be seen \href{https://www.visualcapitalist.com/wp-content/uploads/2019/08/top-100-websites-ranking.html}{here})

To see how some sites evolved in time, one can use \href{https://archive.org/web/}{Wayback Machine} to explore (and/or make) archived versions of sites. 

On the internet, sites can be different not only in time, but by accessing them from different location. (A site (or a server) knows about you \href{https://whatismyipaddress.com/}{this} amount of information.) To access sites, with possible Geo-blocking one can use \href{https://www.howtogeek.com/133680/htg-explains-what-is-a-vpn/}{VPN}.

On the internet there are many-many things, and sometimes you are searching for alternative sites. In these cases \href{https://alternativeto.net/}{AlternativeTo} or \href{https://www.similarsites.com/}{SimilarSites} can be helpful.

Gain information about a website, use \href{http://whois.domaintools.com/}{who is} and possibly check it on \href{https://www.alexa.com/siteinfo/}{Alexa} or on \href{https://www.similarweb.com/}{SimilarWeb}.

\subsection{Current information}

It is crucial to realize, that in general the aim of news organizations is to grab attention, and not to give a balanced world view to the readers.
It does not mean, that news organizations always want to distort our view, but it means, that if someone wants to have a realistic view, then reading the headlines is not enough. Additional sources and statistics are crucial for a better understanding of the world around us.
\begin{itemize}
    \item \href{https://ourworldindata.org/}{Our World in Data}
    \item \href{https://www.gapminder.org/}{Gapminder}
\end{itemize}

To understand the case, one can watch \href{https://www.ted.com/talks/steven_pinker_is_the_world_getting_better_or_worse_a_look_at_the_numbers}{Stephen Pinker}'s presentation, or these articles on \href{https://ourworldindata.org/wrong-about-the-world}{Our World in Data} and \href{https://www.bbc.com/future/article/20190111-seven-reasons-why-the-world-is-improving}{BBC Future} (based on the data from Gapminder).

On the other hand, positive change will not happen, if individuals don't take actions, and past statistics will not inform us about current trends, movements, challenges. For that we need reliable news sources (or a way to extract information from imperfect and biased sources).

To check a news source's \href{https://en.wikipedia.org/wiki/Media_bias}{Media bias} one can use these sites
\begin{itemize}
    \item \href{https://www.allsides.com/media-bias/media-bias-chart}{AllSides}
    \item \href{https://www.adfontesmedia.com/intro-to-the-media-bias-chart/}{The Media Bias Chart}
    \item \href{https://mediabiasfactcheck.com}{Media Bias/Fact Check}
\end{itemize}

To fact check some typically sensational news one can use these sites
\begin{itemize}
    \item \href{https://www.snopes.com/}{Snopes} in English
    \item \href{https://www.urbanlegends.hu/}{Urban legends} in Hungarian
\end{itemize}

An incomplete list of reliable International and/or English news sites:
\begin{itemize}
\item General news with little bias
\begin{itemize}
    \item \href{https://apnews.com/}{AP}
    \item \href{https://www.reuters.com/}{Reuters}
    \item \href{https://www.npr.org/}{NPR}
    \item \href{https://www.bbc.com/news}{BBC News}
    \item \href{https://www.dw.com/en/}{Deutsche Welle}
\end{itemize}
\item Thematic news
\begin{itemize}
    \item \href{https://www.economist.com/}{The Economist} (Economics)
    \item \href{https://www.ft.com/}{Financial Times} (Economics, finance)
    \item \href{https://www.wired.com/}{Wired} (IT, Technology)
    \item \href{https://www.politico.com/}{Politico} (Politics)
    \item \href{https://www.euronews.com/}{EuroNews} (Europe, EU)
    \item \href{https://www.politico.eu/}{Politico Europe} (European Politics)
\end{itemize}
\item Left leaning, but still reliable sources
\begin{itemize}
    \item \href{https://www.theguardian.com/}{The Guardian}
    \item \href{https://www.theatlantic.com/}{The Atlantic}
\end{itemize}
\item Right leaning (economically), but still reliable source \footnote{a more complete USA based list of right wing / conservative news sites is available \href{https://techpresident.com/best-conservative-news-sites/}{here}}
\begin{itemize}
    \item \href{https://reason.com/}{Reason}
\end{itemize}
\end{itemize}

An incomplete list of reliable and/or relevant Hungarian news sites:
\begin{itemize}
    \item \href{https://eduline.hu/}{Eduline} (Education, news about schools and Universities)
    \item \href{https://qubit.hu/}{Qubit} (Science and Technology)
    \item \href{https://telex.hu/}{Telex} (General news, slightly liberal, critical to government)
    \item \href{https://www.valaszonline.hu/}{Válasz} (General, slightly conservative, critical to government)
    \item \href{https://magyarhang.org/}{Magyar Hang} (General, independent, critical to government)
    \item \href{https://merce.hu/}{Mérce} (General, left leaning, critical to government)
    \item \href{https://magyarnemzet.hu/}{Magyar Nemzet} (General, pro government)
    \item \href{https://mandiner.hu/}{Mandiner} (General, pro government)
    \item \href{https://444.hu/}{444} (General, liberal, critical to government)
    \item \href{https://azonnali.hu/}{Azonnali} (General, more or less balanced, critical to government)
    \item \href{https://atlatszo.hu/}{Atlatszó} (Investigatory ``watch dog'' journalism, critical to government)
    \item \href{http://www.transindex.ro/}{Transindex} (General, Hungarian minority in Romania)
    \item \href{https://ujszo.com/}{Ujszo} (General, Hungarian minority in Slovakia)
    \item \href{https://www.magyarszo.rs/}{MagyarSzo} (General, Hungarian minority in Serbia)
\end{itemize}


There is a list of centralized, pro government media sources, called \href{https://cepmf.hu/}{KESMA}, which has a conservative/populist bias. Together with the state media \href{http://mtva.hu/}{MTV}, these news sources represent the governments narrative.

\href{https://index.hu/}{Index} was a slightly liberal portal, critical to government, however in 2020 after changes in top management, most of the journalists resigned, and started \href{https://telex.hu/}{Telex.hu}. After this turn Index become slightly pro government, but kept some elements of its original style.

For a more comprehensive list see the \href{https://en.wikipedia.org/wiki/List_of_newspapers_in_Hungary}{collection of Hungarian newspapers} on Wikipedia.

In politically sensitive questions I highly suggest to read and watch multiple sources, mainly in and about Hungary.

A list of unreliable, ``news'' sites, which produce content purely to maximize their click numbers is published by \href{https://www.urbanlegends.hu/2020/01/megteveszto-magyar-hiroldalak-listaja-2020/}{Urban Legends}.

\subsection{Blogs}

Today everyone with internet connection can be a content creator. Blogging is neither professional journalism, nor scientific publishing, but sometimes professionals and enthusiasts can write about interesting ideas in a accessible way. 

\begin{itemize}
    \item \href{https://medium.com/}{Medium}
\end{itemize}

Examples of informative technical blogs:

\begin{itemize}
    \item \href{https://waitbutwhy.com/}{Wait But Why}
    \item \href{https://www.inference.vc/}{inFERENCe}
    \item \href{https://mathbabe.org/}{mathbabe}
    \item \href{https://jessegalef.com/}{Galef siblings}
\end{itemize}

\subsection{Forums}

Online forums can be good sources of information, if one can critically investigate the opinions and suggestions which can be read there. In general \href{https://stackexchange.com/sites#}{Stack Exchange} is considered as a useful international online forum.
\href{https://www.reddit.com/}{Reddit} is one of the biggest general forum, with many ``subreddits''. The quality of the content is fluctuating, but there are some intersting concepts, like the Ask Me Anything (\href{https://www.reddit.com/r/AMA/}{AMA}) sections, where professionals, interesting and/or popular people answer to questions from the users.
A general forum in the internet is \href{https://www.quora.com/Where-is-the-list-of-all-the-topics-on-Quora}{Quora} this is very general, and further investigation is needed to check the information from it.

In Hungarian one of the oldest IT forums is \href{https://prohardver.hu/forum/index.html}{Prohardver}. A much newer general online forum is \href{https://www.gyakorikerdesek.hu/}{Gyakori kerdesek}, which unfortunately has a low reliability.
There is also a Hungarian subreddit on \href{https://www.reddit.com/r/hungary/}{Reddit}.

\subsection{Misc}

\begin{itemize}
    \item \href{https://www.lesswrong.com/}{Less Wrong}
    \item \href{https://rationality.org/}{Center for Applied Rationality}
    \item \href{https://www.goodreads.com/book/show/656073.The_Skeptic_Encyclopedia_of_Pseudoscience}{The Skeptic Encyclopedia of Pseudoscience}
\end{itemize}

\section{Subjects}

\subsection{Physics}

Physics is one of the ``hardest'' branch in the Natural sciences, and is the purest embodiment of the \href{https://plato.stanford.edu/entries/scientific-method/}{Scientific method}.
It briefly means, that it builds testable mathematical models about the world, mainly in a reductionist way. (Reductionism brought us to the current impressive level of understanding, and predictive power, however, it is less useful in attacking complex systems.)
It has 2-3 branches: Theoretical physics, which is mostly about model making, Experimental physics, which is about collecting data and designing experiments, and Computational physics which is sometimes viewed as a separate branch, and tries to compute the consequences of theoretical models (typically with many constituents).

\begin{itemize}

\item \href{https://www.feynmanlectures.caltech.edu/}{The Feynman Lectures on Physics} One of the best theoretical physics introduction. It uses a little bit of calculus, but because of its didactic style, interested  high school students can read it as well. Feynman on \href{https://www.youtube.com/watch?v=MO0r930Sn_8}{magnets}.

\item \href{https://github.com/sonhuytran/MIT8.01SC.2010F/blob/master/References/University\%20Physics\%20with\%20Modern\%20Physics\%2C\%2013th\%20Edition.pdf}{University Physics with Modern Physics} is a nicely illustrated physics book, it does not use too much higher mathematics. It is also a nice book for interested high school students. (A similar resource is \href{https://powerunit-ju.com/wp-content/uploads/2019/01/Physics-Textbook-9th-E-GearTeam-ilovepdf-compressed.pdf}{Physics for Scientists and Engineers with Modern Physics})

\item \href{https://www.youtube.com/channel/UCiEHVhv0SBMpP75JbzJShqw}{Walter Lewin}. Physics deals with phenomena in Nature. Walter Lewin is a physicist and an entertainer. His demonstrations are simply fun to watch. 

\item \href{http://goliat.eik.bme.hu/~hartlein/}{Härtlein Károly} is a prominent figure of Hungarian physics demonstrations. Explore his demonstrations and physics shows online. His \href{https://hu.wikipedia.org/wiki/H\%C3\%A4rtlein_K\%C3\%A1roly}{wiki} page. (in Hungarian)

\item \href{https://hu.wikipedia.org/wiki/\%C3\%96veges_J\%C3\%B3zsef}{Öveges József}, an iconic Hungarian physics popularizer and teacher. See his \href{https://videakid.hu/videok/termeszet/oveges-professzor-legkedvesebb-kiserleteim-1968-88kk7GyIYkqKKGph}{performance} and/or his \href{https://moly.hu/konyvek/oveges-jozsef-kiserletek-konyve}{book} (on \href{https://www.scribd.com/document/352980511/Oveges-Jozsef-Kiserletek-Konyve-1960}{scribd})

\item \href{https://www.scribd.com/doc/30863917/Negyjegy\%C5\%B1-fuggvenytablazatok}{Négyjegyű} is a Hungarian formula sheet and source of tabulated experimental data.

\item \href{https://en.mandadb.hu/tetel/357528/Fizika}{Szalay Fizika} is a comprehensive handbook about general Physics (from slightly engineering perspective), using high school mathematics. It is in Hungarian. A scanned version of its electromagnetism section can be found \href{https://web.archive.org/web/20200317111146/http://users.atw.hu/gepesz-lev/4felev/fizika.pdf}{here} (see on \href{https://www.scribd.com/document/492897465/Szalay-Fizika}{scribd}).

\item \href{https://www.goodreads.com/book/show/13335561-a-cultural-history-of-physics}{A cultural history of physics} is an extended book, which explores the historical development of Physics, and puts the subject into a wider perspective.

\item Physics competitions:
\begin{itemize}
\item \href{https://www.komal.hu/home.e.shtml}{KöMaL} is a perfect source of challenging problems for high school students from Mathematics, Physics and Computer Science.

\item \href{https://ipho-unofficial.org/}{IPhO} stands for the International Physics Olympiad. Past problems can be found \href{https://physprob.com/}{here}, \href{https://fks.sk/~bzduso/physics/ipho/}{here} and \href{https://omega4edu.org/physics.html}{here}. The syllabus is available \href{https://www.ipho-new.org/statutes-syllabus/}{here}. (past problems in \href{http://ipho.elte.hu/iphos.php}{Hungarian})

\item \href{http://ipho.elte.hu/}{Hungarian IPhO preparation} Seminars, past problems in Hungarian.

\item \href{http://asianphysicsolympiad.org/}{APhO} stands for Asian Physics Olympiad, and has an equal difficulty level to IPhO. Past problems (from 2019) are available \href{https://apho2019.asi.edu.au/resources/past-questions/}{here}.

\item \href{https://physicscup.ee/}{Physics Cup} ``more difficult than the problems of EuPhO, IPhO, and APhO.''

\item \href{http://eik.bme.hu/~vanko/fizika/olimpia.htm}{List} of other international Olympiads

\item \href{http://www.ioaastrophysics.org/}{IOAA} International Olympiad on Astronomy and Astrophysics

\item \href{https://fykos.org/en}{Fykos}

\item \href{https://www.onlinescienceolympiads.org/}{Online Science Olympiads}

\item \href{https://www.iypt.org/}{IYPT} International Young Physicist's Tournament. Team-oriented scientific competition between secondary school students. Hungarian section \href{http://hypt.elte.hu/}{HYPT} and a related competition is \href{http://metal.elte.hu/~icys/}{ICYS}.

\item \href{https://beamlineforschools.cern/home}{BL4S} Beamline for Schools, is an international experiment constructing competition for high school students

\item \href{http://hatvaniverseny.unideb.hu/}{Hatvani István Physics Competition}

\item \href{http://web.archive.org/web/20200318012748/http://fizika.fazekas.hu/index.php/2019/10/01/201920-as-tanev-versenyei-fizikabol/}{List of Hungarian Physics competitions}

\item \href{https://ortvay.elte.hu/main.html}{Rudolf Ortvay International Competition in Physics} University level international physics competition.

\item \href{http://thworldcup.com/about2020}{International Theoretical Physics Olympiad} for Undergraduate Students.

\item \href{http://nyifff.elte.hu/}{NyiFFF} Hungarian outdoor University level (mainly experimental)
Physics competition.

\item \href{https://iptnet.info/}{IPT} International Physicists' Tournament, it is similar to IYPT but for undergraduate University students.
\end{itemize}

\item \href{https://web.archive.org/web/20200318010340/http://hep.ucsb.edu/courses/ph128_18s/Taylor.pdf}{Error Analysis} is a useful basic book to interpret data, collected during experiments. (\href{https://web.archive.org/web/20190801072249/http://www2.phy.ilstu.edu/~wenning/slh/}{Student Laboratory Handbook} gives a similar knowledge.)

\item Softwares for data analysis and visualization
\begin{itemize}
    \item \href{https://www.qtiplot.com/index.html}{QtiPlot} is cross platform (Linux, Mac OS, Windows) ``free'' (but proprietary) plotting and data analysis software. On \href{https://sourceforge.net/projects/qtiplot.berlios/}{Source Forge}, and on \href{https://directory.fsf.org/wiki/Qtiplot}{FSF}.
    \item \href{https://www.originlab.com/}{Origin} is the most popular commercial plotting and data analysis software (works only on Windows)
    \item \href{https://www.wolfram.com/mathematica/}{Wolfram Mathematica} much more general than data visualization and analysis. All the nonlinear fits and the appropriate statistical analysis can be made in this framework as well. (Commercial, cross platform.)
\end{itemize}

\item \href{https://isaacphysics.org/solving_problems}{A few steps during solving problems}

\item \href{https://www.cambridge.org/core/books/200-puzzling-physics-problems/BBD12C169F5FAFDA285F155D62A04D12}{200 Puzzling Physics Problems} creative (non conventional) physics problems. (Available \href{https://dokumen.tips/documents/200-puzzling-physics-problemspdf.html}{online}, but support the authors if the book is useful for you)

\item \href{https://www.cambridge.org/hu/academic/subjects/physics/general-and-classical-physics/200-more-puzzling-physics-problems-hints-and-solutions?format=HB&isbn=9781107103856}{200 More Puzzling Physics Problems} more creative physics problems. (Available \href{https://ia801901.us.archive.org/13/items/200MorePuzzlingPhysicsProblems.pdf1/200\%20More\%20Puzzling\%20Physics\%20Problems.pdf-1.pdf}{online}, but support the authors if the book is useful for you)

\item \href{https://www.typotex.hu/book/8924/gnadig_honyek_vigh_333___furfangos_feladat_fizikabol}{333+ furfangos feladat fizikából} creative physics problems in Hungarian.

\item \href{http://physicstasks.eu/en}{Collection of Solved Problems in Physics} for high school students.

\item \href{https://mek.oszk.hu/14100/14198/14198.pdf}{Kísérleti fizika 1} University level, Hungarian introduction to Physics lecture notes (Mechanics)

\item \href{http://physics.bme.hu/sites/physics.bme.hu/files/users/BMETE12AF46_kov/kisfiz2_jav.pdf}{Kísérleti fizika 2} University level, Hungarian introduction to Physics lecture notes (Electromagnetism and Special relativity)

\item \href{http://physics.bme.hu/sites/physics.bme.hu/files/users/BMETE11AF46_kov/Jegyzet_UO.pdf}{Kísérleti fizika 3} University level, Hungarian  introduction to Physics lecture notes (Thermodynamics and Quantum mechanics)

\item \href{https://en.wikipedia.org/wiki/Course_of_Theoretical_Physics}{Landau-Lifshitz} ``Russian style'' theoretical Physics book for University level

\item \href{https://www.springer.com/gp/book/9780387968902}{Mathematical Methods of Classical Mechanics} classic graduate textbook by mathematician V. I. Arnold. Mathematically accurate and deductive in the same time. (Online available \href{https://loshijosdelagrange.files.wordpress.com/2013/04/v-arnold-mathematical-methods-of-classical-mechanics-1989.pdf}{here})

\item \href{https://www.goodreads.com/book/show/23512.Astronomy_Today}{Astronomy Today} a nice textbook about Astronomy (Another nice online resource can be found on the homepage of \href{https://solarsystem.nasa.gov/basics/}{NASA})

\item \href{https://math.ucr.edu/home/baez/physics/index.html}{Physics FAQ}

\item \href{http://pdg.lbl.gov/}{The Review of Particle Physics} Most recent tabulated data about Particle physics and related parts of Cosmology.

\item \href{https://www.falstad.com/mathphysics.html}{Falstad Math/Physics simulations} Interactive simulations mainly about Electromagnetism and Wave mechanics.

\item \href{https://www.goodreads.com/book/show/1000529.Spacetime_Physics}{Spacetime Physics} by Taylor and Wheeler is a really nice introductory book to relativity (available \href{http://www.gvp.cz/~vinkle/mafynet/_F/str/151621272-Spacetime-Physics-2nd-Ed-Taylor-Wheeler-0716723271.pdf}{here})

\item \href{https://www.spacetimetravel.org/}{Relativistic visualizations}

\item \href{https://www.youtube.com/playlist?list=PLrxfgDEc2NxZJcWcrxH3jyjUUrJlnoyzX}{The Biggest Ideas in the Universe!} by Sean Carroll 

\item \href{http://atomcsill.elte.hu/}{AtomCsill} ``Az atomoktól a csillagokig''. Presentations for high school students in Hungarian.

\item \href{http://felvi.physics.bme.hu/nobeldijas}{Nobel Prize experiments} for high school students on the BUTE (BME). (In Hungarian)

\item \href{https://drustvo-evo.hr/s3/}{$S^3$ and $S^3++$} Science summer schools in Croatia for high school students.

\item \href{http://cuhs.co.uk/wp-content/uploads/2011/12/Szakszavak.pdf}{Hungarian-English dictionary} for scientific terms. (In Hungarian.)

\item Other reading list and books

\begin{itemize}
    \item \href{https://www.youtube.com/watch?v=p9s2fBYA4fU}{Simon Clark}'s list
    \item \href{http://math.ucr.edu/home/baez/books.html}{Baez} list for fundamental/mathematical physics
    \item \href{https://www.goodreads.com/book/show/30359599-forces-of-nature}{Forces of Nature} by Brian Cox
    \item \href{https://www.goodreads.com/book/show/40277241-brief-answers-to-the-big-questions}{Brief Answers to the Big Questions} by Stephen Hawkin
\end{itemize}


\item fun
\begin{itemize}
    \item \href{https://quantum-computing.ibm.com/}{IBM Q-experience} Online quantum computer
    \item \href{https://www.goodreads.com/book/show/21413662-what-if-serious-scientific-answers-to-absurd-hypothetical-questions}{What If?} by Randall Munroe (the crator of \href{https://xkcd.com/}{xkcd}) 
\end{itemize}

\item possible projects:
\begin{itemize}
    \item \href{https://archive.stsci.edu/missions-and-data/tess}{TESS} Transiting Exoplanet Survey Satellite (official \href{https://tess.mit.edu/}{page}) 
\end{itemize}

\end{itemize}




\subsection{Mathematics}

Mathematics has many-many sides, one would say it is a language, others would say it is a playful but rational way of thinking and problem solving, still others would emphasize its extraordinary usefulness in science and technology.

\begin{itemize}
    \item \href{https://moly.hu/konyvek/obadovics-j-gyula-matematika}{Obádovics} book is a detailed and didactic book which is a textbook, perfect for high school students. However, it does not contain too many proofs. Its main aim is to give a solid basis for the usage of high school mathematics mainly in engineering problems. (Which is only one side of mathematics.) It is in Hungarian, online available \href{https://www.scribd.com/document/351926211/Obadovics-J-Gyula-Matematika-pdf}{here}.

    \item \href{https://www.goodreads.com/book/show/192221.How_to_Solve_It}{How to Solve It?} by George Pólya is about how to approach a problem systematically, but creatively. ``Teaching is an Art''. See a summary \href{https://math.berkeley.edu/~gmelvin/polya.pdf}{here} (or on \href{https://en.wikipedia.org/wiki/How_to_Solve_It}{wikipedia}), and a Pólya's performance \href{https://www.youtube.com/watch?v=h0gbw-Ur_do}{here}.
    
    \item \href{https://www.goodreads.com/book/show/696238.Proofs_from_THE_BOOK}{Proofs from THE BOOK} an idea of Paul Erdős, where all the beautiful proofs are listed (``The Book'' is only accessible for ``God'', we can only construct imperfect versions.) A \href{https://www.youtube.com/watch?v=-oxfHwSzoM4}{lecture} of Erdős in Hungarian.
    
    \item Handbooks
    \begin{itemize}

        \item \href{https://www.goodreads.com/book/show/1904487.Handbook_of_Mathematics}{Bronshtein and Semendyayev} An extensive ``handbook'' of mathematical formulas
    
        \item \href{https://www.goodreads.com/book/show/1296073.Handbook_of_Mathematical_Functions}{Handbook of Mathematical Functions} by Abramowitz. An extensive handbook of (special) functions and formulas.
    
        \item \href{https://dlmf.nist.gov/}{Online Handbook} of functions and formulas.
    \end{itemize}
    
    \item More about proofs
    \begin{itemize}
        \item \href{https://www.goodreads.com/book/show/365666.Proofs_Without_Words}{Proofs Without Words} visual ``proofs'' give an intuitive understanding of various theorems. Follow up books \href{https://www.goodreads.com/book/show/365667.Proofs_Without_Words_II}{here} and \href{https://www.goodreads.com/book/show/9655655-charming-proofs}{here}.
    
        \item \href{https://www.goodreads.com/book/show/739735.How_to_Prove_It}{How to Prove It} ``The transition from solving problems to proving theorems''
    
        \item \href{http://farside.ph.utexas.edu/Books/Euclid/Elements.pdf}{Elements} by Euclid
    \end{itemize}
    
    \item Competitions/Fun
    \begin{itemize}
        \item \href{https://www.komal.hu/home.e.shtml}{KöMaL} is a perfect source of challenging problems for high school students from Mathematics, Physics and Computer Science.
        
        \item \href{https://agondolkodasorome.hu/}{A gondolkodás öröme} Exploratory experiencing of Mathematics (in Hungarian)
        
        \item \href{https://erdosiskola.mik.uni-pannon.hu/}{Erdős Mathematics School}
        
        \item \href{https://agondolkodasorome.hu/versenynaptar/}{Calendar for Hungarian Competitions} (and another \href{http://matek.berzsenyi.hu/versenynaptar}{calendar}.)
        
        \item \href{https://molympiad.wordpress.com/list-of-mathematics-competitions/}{List of mathematics competitions} world wide
        
        \item \href{https://www.imo-official.org/}{IMO} International Mathematical Olympiad
        
        \item \href{https://www.egmo.org/}{European Girls’ Mathematical Olympiad}
        
        \item \href{https://memo-official.org/MEMO/contests/previous/}{MEMO} Middle European Mathematical Olympiad. (Hungarian \href{https://memo.szolda.hu/}{homepage})
        
        \item \href{http://www.bolyai.hu/aranydaniel.htm}{Arany Dániel} Competition (Hungarian)
        
        \item \href{http://www.bolyai.hu/kurschak.htm}{Kürschák József} Competition (Hungarian)
        
        \item \href{http://www.bolyai.hu/schweitzer.htm}{Schweitzer Miklós} Competition (Hungarian)
        
        \item \href{http://www.bolyaiverseny.hu/index.html}{Bolyai competition} Group competition Hungarian, and International.
        
        \item \href{http://www.mategye.hu/}{MaTeGyE} further math competitions in Hungary
        
        \item \href{https://math.naboj.org/}{Náboj}
        
        \item \href{https://www.maa.org/math-competitions/putnam-competition}{Putnam Competition} for USA and Canadian undergraduate students.
        
        \item \href{http://kvant.mccme.ru/}{Квант} magazine in Russian. (Archived \href{https://www.nsta.org/publications/quantum.aspx}{Quantum} issues in English)
        
        \item \href{https://www.wycombeabbey.com/the-dove-125-mathematics-competition/}{Wycombe Abbey Summer Mathematics Competition}
        
        \item University level:
        \begin{itemize}
            \item \href{https://www.imc-math.org.uk/}{International Mathematics Competition for University Students}
        \end{itemize}
        
    \end{itemize}
    
    \item \href{https://www.goodreads.com/book/show/112243.Concrete_Mathematics}{Concrete mathematics: a foundation for computer science}
    
    \item Linear Algebra
    \begin{itemize}
        \item \href{https://www.scolar.hu/vektoralgebra_matrixok_determinansok_tobbvaltozos_fuggvenyek_1820}{Obádovics} Vektoralgebra; mátrix determinánsok; többváltozós függvények. A basic, application based introduction in Hungarian
        
        \item \href{https://www.youtube.com/playlist?list=PLZHQObOWTQDPD3MizzM2xVFitgF8hE_ab}{Linear Algebra} by 3Blue1Brown
        
        
        \item \href{https://web.archive.org/web/20160717031653/http://tankonyvtar.ttk.bme.hu/pdf/14.pdf}{Linear Algebra} A nicely illustrated ``second course'' in Hungarian
        
        \item \href{https://www.goodreads.com/book/show/309768.Linear_Algebra_Done_Right}{Linear Algebra Done Right} ``Perfect second textbook'' for linear algebra
    \end{itemize}
    
    \item Calculus
    \begin{itemize}
        \item \href{https://moly.hu/konyvek/obadovics-j-gyula-szarka-zoltan-felsobb-matematika}{Obádovics: Felsőbb matematika} is high quality didactic textbook, focusing on the usage and application of Calculus. (In Hungarian.) \href{https://www.scribd.com/doc/96616339/Obadovics-J-Gyula-Fels\%C5\%91bb-matematika}{Online} available.
        
        \item \href{https://www.youtube.com/watch?v=WUvTyaaNkzM&list=PLZHQObOWTQDMsr9K-rj53DwVRMYO3t5Yr}{Essence of Calculus} on YouTube
        
        \item Two standard calculus books with problems from \href{https://www.goodreads.com/book/show/61298.Calculus}{Stewart} and \href{https://www.goodreads.com/book/show/328645.Calculus}{Spivak}
        
        \item \href{https://www.goodreads.com/book/show/292079.Principles_of_Mathematical_Analysis}{Principles of Mathematical Analysis} from Walter Rudin is a classic Mathematical introductory to Calculus also know as Mathematical Analysis. It focuses on the coherent mathematical foundation of calculus, and on applications.
        
        \item \href{https://www.goodreads.com/book/show/502785.Analysis_I}{Analysis I}, \href{https://www.goodreads.com/book/show/2578776-analysis-ii}{Analysis II} by Terence Tao
        
        \item Counterexamples
            \begin{itemize}
            \item \href{https://www.goodreads.com/book/show/1402795.Counterexamples_in_Probability_and_Real_Analysis}{Counterexamples in Probability and Real Analysis}
        
            \item \href{https://www.goodreads.com/en/book/show/818070.Counterexamples_in_Analysis}{Counterexamples in Analysis}
        
            \item \href{https://www.goodreads.com/book/show/116419.Counterexamples_in_Topology}{Counterexamples in Topology}
        
            \item \href{https://www.goodreads.com/book/show/2106285.Counterexamples_in_Probability}{Counterexamples in Probability}
    
    \end{itemize}
        
    \end{itemize}
    
    \item Mathematical softwares
    \begin{itemize}
        \item \href{https://www.wolfram.com/mathematica/}{Wolfram Mathematica} mainly for symbolic computations (commercial)
        
        \item \href{https://www.sagemath.org/}{Sage} an Open source but less powerful alternative for simbolic calculations
        
        \item \href{https://www.geogebra.org/}{GeoGebra} is a nice tool for geometry problems
        
        Further softwares
        \begin{itemize}
            \item \href{https://www.mathworks.com/products/matlab.html}{Matlab} is useful mainly for numerical analysis
        
            \item \href{https://www.gnu.org/software/octave/}{GNU Octave} is a useful open source alternative for Matlab
        
            \item \href{http://www.gnuplot.info/}{gnuplot} a very basic open source plotting and fitting tool.
        
            \item \href{https://www.r-project.org/}{R} used typically for statistics
        
            \item \href{https://jupyter.org/}{Jupiter} is a nice exploring and presenting tool together with (for instance) \href{https://www.python.org/}{Python}.
            
            \item much more can be found on Wikipedia, starting for instance from \href{https://en.wikipedia.org/wiki/List_of_open-source_software_for_mathematics}{here}.
        \end{itemize}
    \end{itemize}
    
    \item Comprehensive textbooks
    \begin{itemize}
        \item \href{https://www.goodreads.com/book/show/584620.What_Is_Mathematics_}{What is Mathematics} ``Mathematics as an expression of the human mind reflects the active will, the contemplative reason, and the desire for aesthetic perfection.''
        
        \item \href{https://www.goodreads.com/book/show/405880.Mathematics}{Mathematics: Its Content, Methods and Meaning} close to real world but precise and high quality comprehensive book in the ``Russian spirit'' familiar from the books of Arnold and Kolmogorov.
        
        \item \href{https://www.goodreads.com/book/show/1471873.The_Princeton_Companion_to_Mathematics}{The Princeton Companion to Mathematics} ``General Map of Mathematical World''
    \end{itemize}
    
    \item History of Mathematics
    \begin{itemize}
        \item \href{https://www.goodreads.com/book/show/786570.A_History_of_Mathematics}{History of Mathematics}
        \item \href{https://mek.oszk.hu/05000/05052/}{Nincs királyi út!} by Sain Márton. History of Mathematics in Hungarian
        \item \href{https://soundcloud.com/stephenwolfram/a-very-brief-history-of-mathematics}{A Very Brief History of Mathematics} podcast by Stephen Wolfram
        \item \href{https://genealogy.math.ndsu.nodak.edu/index.php}{Math Geneology Project}
        \item \href{https://history-of-mathematics.org/}{History of Mathematics Project}
    \end{itemize}
    
    
    
    
    
    
    
    \item \href{https://www.mateking.hu/}{Mateking} partially commercial, but useful source of study materials (Hungarian).
    
    \item \href{http://www.cimt.org.uk/projects/mep/index.htm}{Mathematics Enhancement Programme} Study materials from The Centre for Innovation in Mathematics Teaching \href{http://www.cimt.org.uk/}{CIMT}.
    
    \item \href{https://artofproblemsolving.com/online}{Art of Problem Solving} commercial, but nice.
    
    \item \href{https://www.goodreads.com/book/show/265415.Solving_Mathematical_Problems}{Solving Mathematical Problems} by Terence Tao
    
    \item \href{https://terrytao.wordpress.com/}{Terence Tao} 
    
    \item \href{http://www.claymath.org/millennium-problems}{Millenium problems} Unsolved problems in Mathematics
    
    \item \href{https://mathoverflow.net/questions/66084/open-problems-with-monetary-rewards}{Open Problems With Monetary Rewards}
    
    \item \href{https://matek.fazekas.hu/index.php?option=com_content&view=article&id=156&Itemid=217}{Hungarian final examination} (Érettségi feladatok) in Hungarian
    
    \item Other books and sources
    \begin{itemize}
        \item \href{https://web.evanchen.cc/geombook.html}{Euclidean Geometry in Mathematical Olympiads} by Evan Chen
        \item \href{https://web.evanchen.cc/napkin.html}{An Infinitely Large Napkin}
        \item \href{https://publicdomainreview.org/collection/the-first-six-books-of-the-elements-of-euclid-1847}{Illustrated Elements of Euclid} by Oliver Byrne 
    \end{itemize}
    
    \item Other reading lists
    \begin{itemize}
        \item \href{https://www.maths.cam.ac.uk/documents/reading-list.pdf/}{Cambridge Mathematical reading list}
        \item \href{https://www.maths.ox.ac.uk/study-here/undergraduate-study/prospectus}{Oxford reading list} under ``Recommended Mathematics Reading''
        \item \href{https://math.mit.edu/research/highschool/primes/reading.php}{MIT reading list}
        \item \href{http://math.ucr.edu/home/baez/books.html}{Baez} list for fundamental/mathematical physics
        \item \href{https://www.youtube.com/watch?v=I_Df_mx8Hxo}{Toby's list}
    \end{itemize}
    
    
\end{itemize}

\subsection{Computer Science}

Computer Science (CS) is a huge umbrella term, which covers Algorithms, Programming, Theoretical information theory, Networking, Cybersecurity, Artificial Intelligence (AI), Robotics and vast number of other areas.
Broadly speaking it can be divided into Theoretical Computer Science, and Applied Computer Science, which is basically programming.

This branch is more industry and application driven then Physics and Mathematics in general, because of its huge and recent impact on society (see \href{https://en.wikipedia.org/wiki/Digital_Revolution}{Digital revolution} and the  \href{https://www.weforum.org/agenda/2016/01/the-fourth-industrial-revolution-what-it-means-and-how-to-respond/}{fourth industrial revolution}).
However, many other academic and even ethical questions arise which are deeply connected to Computer Science and its applications.

Computer science is a rapidly advancing field and the cutting edge is moving very fast. Because of that computer science is not something what can be learned completely from an older, more experienced teacher, but is a constantly ongoing discourse, between accumulated knowledge and novel technologies.

Because of its reflexivity on society, \href{https://en.wikipedia.org/wiki/Ethics_of_artificial_intelligence}{AI ethics} is a relevant part of CS.


\begin{itemize}
    
    \item \href{https://www.cs.cmu.edu/~15110-s13/Wing06-ct.pdf}{Computational Thinking}
    
    \item \href{https://www.youtube.com/watch?v=SzJ46YA_RaA}{Map of Computer Science} shows the current domain of CS (\href{https://i.pinimg.com/originals/b0/d1/da/b0d1da585ebcac73f17e10e91b30eaf6.png}{image} format)

    \item \href{https://www.cambridgeinternational.org/programmes-and-qualifications/cambridge-international-as-and-a-level-computer-science-9608/}{Cambridge A-level} gives a summary about various topics, covering basic skills and knowledge
    \begin{itemize}
        \item \href{https://www.goodreads.com/book/show/13725701-cambridge-international-as-and-a-level-computing-coursebook}{CS A-level Course book}. Online available \href{https://web.archive.org/web/20170711125358/https://www.gceguide.xyz/files/e-books/a-level/Computing.pdf}{here} and \href{https://web.archive.org/web/20210228062647/https://www.gceguide.xyz/gce-e-books/cambridge-international-as-a-level-computer-science-coursebook}{here}.
    \end{itemize}
     
    \item Books about Problem solving/Computational thinking/CS in general:
    \begin{itemize}

        \item \href{https://www.goodreads.com/book/show/3086628-how-to-solve-it-by-computer}{How to Solve it by Computer}

    
        \item \href{https://www.goodreads.com/book/show/964709.The_New_Turing_Omnibus}{The New Turing Omnibus} is an exhaustive review of topics in CS (suggested for freshmans on CS at Cambridge)
    
        \item \href{https://www.goodreads.com/book/show/6220915-how-to-think-like-a-mathematician}{How to think like a mathematician} helps to get familiar with analytic thinking required for CS (suggested for freshmans on CS at Cambridge)
    
    \end{itemize}
    
    \item \href{https://www.codecademy.com/}{Codecademy} is a useful place to learn programming
    
    \item \href{https://www.tutorialspoint.com/computer_programming_tutorials.htm}{tutorialspoint} provides quick tutorials for many languages, together with an online compiler for the code.

    
    \item Programming languages. There is a plethora of programming \href{https://www.levenez.com/lang/}{languages}, however here I will list a few from the \href{http://rigaux.org/language-study/diagram-light.pdf}{``more relevant''} ones. From an even shorter \href{https://www.computerscience.org/resources/computer-programming-languages/}{list} I would mention a few (another measure of popularity can be seen \href{https://www.tiobe.com/tiobe-index/}{here}.)
    \begin{itemize}
        \item \href{https://www.youtube.com/watch?v=poJfwre2PIs}{Overview of languages} from job market point of view
        \item \href{https://en.wikipedia.org/wiki/C_(programming_language)}{C} basic Procedural language 
        \item \href{https://isocpp.org/}{C++} basic Object-Oriented language
        \item \href{https://www.python.org/}{Python} a very popular, easy to learn language
        \item other languages
        \begin{itemize}
        \item \href{https://en.wikipedia.org/wiki/HTML}{HTML} Hypertext Markup Language, the ``language of websites''
        \item \href{https://en.wikipedia.org/wiki/SQL}{SQL} for standard database management
        \item \href{https://www.latex-project.org/}{LaTeX} markup language of most academic papers
        \item \href{https://en.wikipedia.org/wiki/Java_(programming_language)}{Java} is another very popular programming language, which due to running on a Virtual Machine, is very suitable for developing cross platform applications (such as different operating systems, e.g., notably Android).
        \item \href{https://www.gnu.org/software/bash/}{bash} scripting language
        \item \href{https://www.haskell.org/}{Haskell} a popular functional programming language
        \end{itemize}
    \end{itemize}
    
    \item Hello World. Learning a new programming language starts with the creation of a tiny working code, usually printing or outputing ``Hello World''.  The first working code is crucial, because from it one can explore the functionalities of a language (or a package) by little, incremental steps trough working codes. (Without a complete list of languages and editors, or Integrated Development Environments (IDE-s).)
    \begin{itemize}
        \item \href{https://www.youtube.com/watch?v=87u667BoHkg}{HW} C, Linux (Ubuntu).
        \item \href{https://www.youtube.com/watch?v=2NWeucMKrLI}{HW 01}, \href{https://www.youtube.com/watch?v=3DeLiClDd04}{02}, \href{https://www.youtube.com/watch?v=iWx3yyFMWQA}{03}, \href{https://www.youtube.com/watch?v=oSpmApiUsHw}{04} C, Windows. The tutorial in one \href{https://www.youtube.com/playlist?list=PL6gx4Cwl9DGAKIXv8Yr6nhGJ9Vlcjyymq}{HW list}.
        \item \href{https://www.youtube.com/watch?v=98fYq1BIOEk}{HW} Python, Linux (Ubuntu)
        \item \href{https://www.youtube.com/watch?v=KOdfpbnWLVo}{HW} Python cross platform
        \item \href{https://www.youtube.com/watch?v=yXMb7SC9gHg}{HW} C++, Linux (Ubuntu)
        \item \href{https://www.youtube.com/watch?v=VBjEoGX5rFI}{HW} C++, Windows
        \item \href{http://rosettacode.org/wiki/Hello_world/Text}{Hello World} code on many-many other programming languages. (Another collection \href{http://helloworldcollection.de/}{here})
    \end{itemize}
    
    \item Standard books on programming
    \begin{itemize}
        \item \href{https://www.goodreads.com/book/show/120642.C_Primer_Plus}{C Primer Plus} A huge, but popular introductory book into C
        \item \href{https://www.goodreads.com/book/show/515601.The_C_Programming_Language}{The C     programming language} by Kernighan and Ritchie. A classic, from the creators of the language.
        
        \item \href{https://moly.hu/konyvek/pere-laszlo-unix-gnu-linux-programozas-c-nyelven}{Programozás C nyelven} by Pere László, a book in Hungarian (online available \href{https://people.inf.elte.hu/budraai/Prog/c\%20prog.pdf}{here})
        
        \item \href{https://www.goodreads.com/book/show/768080.C_Primer}{C++ Primer} A huge, but popular introductory book into C++
        \item \href{https://www.goodreads.com/book/show/112251.The_C_Programming_Language}{C++} by Bjarne Stroustrup. A classic, from the creators of the language.
        \item \href{https://www.goodreads.com/book/show/475667.An_Introduction_to_Python}{An Introduction to Python} an introduction from the creator, Guido van Rossum.
        \item \href{https://diveintopython3.problemsolving.io/}{Dive Into Python 3} introduction to Python, for those, who already know some programming
        
        %%% Pyhon book and links for ppl who alreayd know programming:
        % python konyv: http://diveintopython3.ep.io/
        % numpy es scipy, a matlab alapfunkciokhoz (vektorok, matrixok, ...):
        % http://numpy.scipy.org/
        % http://www.scipy.org/
        % matplotlib a plottolashoz: http://matplotlib.sourceforge.net/
        % ipython: matlab szeru IDE+elosztott futtatas:
        % http://ipython.org/ipython-doc/rel-0.12/index.html
        % scikit-learn: machine learning toolbox:
        % http://scikit-learn.org/stable/
        % egyeb toolboxok:
        % http://scikits.appspot.com/scikits
        % SAGE: python alapu matek rendszer, nagyon sokminden van benne, ill.
        % matlabot lehet hivni belole:
        % http://www.sagemath.org/
        % Matlabrol numpy-re (pythonra) attereshez segitseg:
        % http://www.scipy.org/NumPy_for_Matlab_Users

        \item \href{https://www.goodreads.com/book/show/3735293-clean-code}{Clean Code} for an easy (not impossible) to read coding style
        
        \item \href{https://arxiv.org/pdf/1210.0530.pdf}{Best Practices for Scientific Computing}
    \end{itemize}
    
    \item Graphics
    \begin{itemize}
        \item \href{https://www.opengl.org/}{OpenGL} 3D/2D graphics
        \item \href{https://www.libsdl.org/}{SDL} is a popular library
        \item \href{https://unity.com/}{Unity} a popular cross-platform Game engine
        \item \href{https://liballeg.org/}{Allegro} is simple to use library for 2D graphics
    \end{itemize}
    
    
    \item Integrated Development Environment (IDE). In practice it is good to have an environment, which is essentially a text editor and a compiler, in which one can write, run, debug, and make a version control of its programs. Here are some good to know IDE-s
    \begin{itemize}
        \item \href{https://codelite.org/}{Code Lite} C/C++ IDE
        \item \href{https://www.spyder-ide.org/}{Spyder} Python IDE
        \item \href{https://www.jetbrains.com/pycharm/}{PyCharm} Python IDE
        \item other notable IDE-s
        \begin{itemize}
            \item \href{https://www.sublimetext.com/}{Sublmie text}
            \item \href{https://www.codeblocks.org/}{Code::Blocks}
            \item \href{https://visualstudio.microsoft.com/}{Visual Studio}
            \item \href{https://www.eclipse.org/ide/}{Eclipse}
        \end{itemize}
    \end{itemize}
    
    \item Online references
    \begin{itemize}
        \item \href{http://www.cplusplus.com/doc/tutorial/}{C++ reference}
        \item \href{http://rosettacode.org/wiki/Rosetta_Code}{Rosetta Code} example programs in a huge variety of languages
        \item \href{https://www.w3schools.com/html/}{html}
        \item \href{https://www.w3schools.com/}{w3schools} for web development in general (HTML, PHP, SQL, CSS, JavaScript, Python, etc.)
    
        \item bash
        \begin{itemize}
            \item \href{https://www.howtogeek.com/140679/beginner-geek-how-to-start-using-the-linux-terminal/}{Basic introduction to terminal}, and other useful \href{https://itsfoss.com/linux-command-tricks/}{tricks}.
            \item \href{https://tldp.org/HOWTO/Bash-Prog-Intro-HOWTO.html}{How To}
            \item \href{https://www.tldp.org/LDP/Bash-Beginners-Guide/html/}{Beginner Guide}
            \item \href{https://tldp.org/LDP/abs/html/}{Advanced Guide}
            \item \href{https://devhints.io/bash}{Cheatsheets} 
        \end{itemize}
        
        \item \href{http://lambda.inf.elte.hu}{ELTE Haskell} interactive university course (in Hungarian).
    \end{itemize}
    
    \item Libraries
    \begin{itemize}
        \item \href{https://www.gnu.org/software/gsl/}{GSL} GNU Scientific Library (for C, C++) \href{https://en.wikipedia.org/wiki/GNU_Scientific_Library}{Wiki}
        \item \href{https://www.boost.org/}{Boost} a set of libraries for the C++
        \item \href{https://numpy.org/}{NumPy} is the fundamental package for scientific computing with Python (a \href{https://docs.scipy.org/doc/numpy/user/numpy-for-matlab-users.html}{tutorial} for those, who are already familiar with Matlab)
        \item \href{https://www.scipy.org/}{SciPy} Scientific computing library for Python
        \item \href{https://matplotlib.org/}{Matplotlib} plotting and visualization library for Python
        \item \href{https://www.anaconda.com/products/individual}{Anaconda} a whole (mostly Python) environment for Data Science
    \end{itemize}
    
    \item Operational Systems (OS) (Popularity of OS-s can be found \href{https://www.w3schools.com/browsers/browsers_os.asp}{here} and \href{https://gs.statcounter.com/os-market-share}{here})
    \begin{itemize}
        \item \href{https://www.virtualbox.org/}{Virtual Box} is a very useful emulator to try out different OS-s in a virtualized environment
        
        \item \href{https://en.wikipedia.org/wiki/Linux}{Linux} which ha many \href{https://en.wikipedia.org/wiki/Linux_distribution}{distributions}.
        \begin{itemize}
            \item \href{https://ubuntu.com/}{Ubuntu} is useful for general use. (A guide for \href{https://itsfoss.com/install-ubuntu/}{installation})
            \item \href{https://www.kali.org/}{Kali} is a ``hacking OS'' with many built in Penetration Testing tools
        \end{itemize}
        \item \href{https://www.apple.com/macos/}{MacOS} is a Unix based OS of apple computers
        \item \href{https://www.microsoft.com/en-us/windows/}{Windows} originally an MS-DOS based operating system
        \item \href{https://www.android.com/}{Android} is a Linux based mobile OS
        \item \href{https://www.apple.com/ios/}{iOS} is a Unix like OS for mobiles
    \end{itemize}
    
    \item useful shell programs
    \begin{itemize}
        \item \href{https://linux.die.net/man/1/ssh}{ssh} (see this  \href{https://www.howtogeek.com/311287/how-to-connect-to-an-ssh-server-from-windows-macos-or-linux/}{how to} for more detail)
        \item \href{https://www.gnu.org/software/wget/manual/wget.html}{wget}
        \item \href{https://linux.die.net/man/1/rsync}{rsync}
        \item \href{https://linux.die.net/man/1/scp}{scp}
        \item \href{https://linux.die.net/man/1/htop}{htop}
        \item \href{https://linux.die.net/man/1/mc}{mc} Midnight Commander
    \end{itemize}
    
    \item other useful applications
    \begin{itemize}
        \item \href{https://www.wireshark.org/}{Wireshark}
        \item \href{https://www.qbittorrent.org/}{qBittorent}
        \item \href{https://mega.nz/}{MEGA}
    \end{itemize}
    
    \item \href{https://git-scm.com/}{git} and \href{https://github.com/}{GitHub}: essential for version control and collaborative development
    \begin{itemize}
        \item \href{https://www.youtube.com/watch?v=SWYqp7iY_Tc}{Git \& GitHub} for Windows
        \item \href{https://www.youtube.com/watch?v=USjZcfj8yxE}{git}, \href{https://www.youtube.com/watch?v=nhNq2kIvi9s}{GitHub} for Linux (Ubuntu)
        \item \href{https://git-scm.com/book/en/v2}{Pro Git book}
        \item \href{https://about.gitlab.com/}{GitLab} an open source alternative to GitHub
    \end{itemize}
    
    \item Books and resources on Theoretical Computer Science
    \begin{itemize}
    
        \item \href{https://www.goodreads.com/book/show/10803540-algorithms}{Algorithms} (another \href{https://www.goodreads.com/book/show/108986.Introduction_to_Algorithms}{book})

        \item \href{https://cs.stanford.edu/~knuth/taocp.html}{Art of Computer Programming}

        \item \href{https://cs.stanford.edu/~knuth/}{Knuth}
        
        \item \href{https://www.goodreads.com/book/show/1556746.Computability_and_Logic}{Computability and Logic}
        
        \item \href{https://www.goodreads.com/book/show/433439.Elements_of_Information_Theory}{Information theory}
    
    \item AI
    \begin{itemize}
        \item \href{https://www.andrewng.org/}{Andrew Ng} and his courses on \href{https://www.coursera.org/instructor/andrewng}{Coursera}
        \item \href{https://www.goodreads.com/book/show/201357.Information_Theory_Inference_and_Learning_Algorithms}{MacKay}'s book
        \item \href{https://www.goodreads.com/book/show/55881.Pattern_Recognition_and_Machine_Learning}{Bishop}
        \item \href{https://www.goodreads.com/book/show/24072897-deep-learning}{Deep Learning}
        \item \href{https://www.goodreads.com/book/show/739791.Reinforcement_Learning}{Reinforcement Learning}
        \item \href{https://tor-lattimore.com/downloads/book/book.pdf}{Bandit book}
        \item \href{https://deepmind.com/learning-resources/-introduction-reinforcement-learning-david-silver}{Introduction Reinforcement Learning} lectures by David Silver in \href{https://deepmind.com/}{DeepMind} and \href{https://www.ucl.ac.uk/}{UCL}
        \item \href{https://www.youtube.com/channel/UCBOgpkDhQuYeVVjuzS5Wtxw}{Virtual Machine Learning Summer School 2020}
    \end{itemize}

    \end{itemize}
    
    
    \item Other reading lists
    \begin{itemize}
        \item \href{https://intelligence.org/research-guide/}{MIRI} reading list
        \item \href{https://ai.stackexchange.com/questions/3374/how-does-one-start-learning-artificial-intelligence}{AI Stackexchange}
        \item \href{https://cs.stackexchange.com/questions/59319/best-starter-resources-for-learning-about-ai}{CS Stackexchange}
        \item \href{https://www.cs.ox.ac.uk/admissions/undergraduate/why_oxford/background_reading.html}{Oxford CS reading list}
        \item \href{http://www.cs.cmu.edu/~./books/}{Carnegie Mellon} list of books
    \end{itemize}
    
    
    \item Competitions/heckatons
    \begin{itemize}
        \item \href{https://www.komal.hu/home.e.shtml}{KöMaL} is a perfect source of challenging problems for high school students from Mathematics, Physics and Computer Science.
        \item \href{https://ioinformatics.org/}{IOI} International Olympiad in Informatics
        \item \href{http://ceoi.inf.elte.hu/}{Central European Olympiad of Informatics}
        \item \href{https://codingcompetitions.withgoogle.com/}{Google's coding competitions}
        \item \href{https://codeforces.com/}{Codeforces}
        \item \href{http://informatika.fazekas.hu/informatika-versenyek-201920/}{Hungarian competitions}
        \item \href{https://www.mycplus.com/featured-articles/programming-contests-and-challenges/}{List of competitions}
        \item University level
        \begin{itemize}
            \item \href{https://icpc.baylor.edu/}{ICPC} International Collegiate Programming Contest. (A nice preparation book is \href{https://www.goodreads.com/book/show/22820951-competitive-programming}{Competitive Programming})
            \item \href{http://www.challenge24.org/}{Challenge 24}
            \item \href{https://www.kaggle.com/}{Kaggle} Data Science challenges
        \end{itemize}
        \item Robotics
        \begin{itemize}
            \item \href{https://www.robocup.org/}{Robocup}
            \item \href{https://robotprog.hu/}{Robotverseny}
            \item \href{https://www.microbot.hu/}{micro:bot}
        \end{itemize}
    \end{itemize}
    
   \item fun
   \begin{itemize}
        \item \href{https://www.codingame.com/}{Codingame}
        \item \href{https://lightbot.com/}{Lightbot}
   \end{itemize}
   \item Quantum computing
   \begin{itemize}
       \item \href{https://qiskit.org/}{Qiskit} IBM’s Python-based library
   \end{itemize}

    \item Robotics
    \begin{itemize}
        \item \href{https://www.arduino.cc/}{Arduino}
        \item \href{https://www.makerspace.hu/}{Maker Space} in Hungary
    \end{itemize}
    
    \item 3D design and CAD
    \begin{itemize}
        \item \href{https://en.wikipedia.org/wiki/AutoCAD}{Auto CAD}, for other CAD softwares see this \href{https://all3dp.com/1/best-cad-software/}{list}
    \end{itemize}
    
    \item Ethical hacking
    \begin{itemize}
        \item \href{https://www.hackerone.com/}{HackerOne}
        \item \href{https://defcon.org/}{Defcon}
    \end{itemize}

    \item possible projects
    \begin{itemize}
        \item \href{https://techtabor.agondolkodasorome.hu/}{Techtábor} (in Hungary)
        \item \href{https://www.skawa.hu/}{Skawa} startup projects in Hungary
    \end{itemize}
\end{itemize}

\subsection{Chemistry}

\textbf{Disclaimer: I'm not a chemist, and didn't do any serious chemistry during my University studies. My view on chemistry is highly subjective, and probably naive. A contribution of a more experienced chemist would be valuable.}

I think chemistry as a field is worthless without experiments. If I needed to introduce chemistry to a high school student, I would say, that it is primarily magic and craft, and its view from a analytical and critical lens is secondary (however that makes chemistry a science).



\begin{itemize}
    \item Demonstrations by \href{https://www.youtube.com/playlist?list=PLbnrZHfNEDZxPZ369tAF0wjnNo-A3EcDi}{Andrew Szydlo} and \href{https://www.youtube.com/watch?v=ti_E2ZKZpC4}{Chris Bishop}.
    \item \href{https://www.youtube.com/channel/UCFhXFikryT4aFcLkLw2LBLA}{Fun and interesting videos} by NileRed 
    
    \item \href{https://www.youtube.com/watch?v=P3RXtoYCW4M}{Map of Chemistry}
    
    
    \item Experiments
    \begin{itemize}
        \item \href{https://melscience.com/US-en/articles/safety-guide-conducting-chemical-reactions-home/}{Safety}
        \item \href{https://melscience.com/US-en/experiments/}{Experiments at home}
    \end{itemize}
    
    \item Laboratory experiments
    \begin{itemize}
        \item \href{https://www.youtube.com/watch?v=9o77QEeM-68}{Safety} (less detailed \href{https://www.youtube.com/watch?v=VRWRmIEHr3A}{version})
        \item \href{https://www.youtube.com/watch?v=6F6D3XuZDVo&list=PLC10F4535D849964F}{AP Chemistry}
    \end{itemize}
    
    \item \href{https://www.cambridgeinternational.org/programmes-and-qualifications/cambridge-international-as-and-a-level-chemistry-9701/}{Cambridge International AS and A Level Chemistry}
    \begin{itemize}
        \item \href{https://www.cambridge.org/gb/education/subject/science/chemistry/cambridge-international-a-level-chemistry-2nd-edition/cambridge-international-as-and-a-level-chemistry-2nd-edition-coursebook-cd-rom?isbn=9781107638457&format=WW}{Course book} online available \href{https://web.archive.org/web/20210411103136/https://www.gceguide.xyz/gce-e-books/cambridge-international-as-and-a-level-chemistry-coursebook-2nd-edition}{here}
    \end{itemize}
    
    \item \href{https://bookline.hu/product/home.action?_v=Naray_Szabo_Gabor_Kemia&type=22&id=48903}{Náray-Szabó} Chemistry book in Hungarian (online available \href{https://www.scribd.com/doc/249972941/Kemia-Naray-Szabo-Gabor}{here})
    
    \item \href{https://www.goodreads.com/author/show/7164718.Peter_Atkins}{Atkins} books are generally suggested for University chemistry courses
    
    \item \href{https://chemistry.stackexchange.com/questions/37303/resources-for-learning-chemistry}{Resources for Learning Chemistry}
    
    \item \href{https://www.youtube.com/channel/UC0cd_-e49hZpWLH3UIwoWRA}{Professor Dave Explains} YT channel
    
    \item \href{https://www.youtube.com/user/AmerChemSoc/playlists?view=50&sort=dd&shelf_id=16}{American Chemical Society}'s YouTube channel, to find out what chemists do.
    
    \item \href{https://www.mke.org.hu/index.php}{Magyar Kémikusok Egyesülete}
    
    \item \href{https://www.rsc.org/}{Royal Society of Chemistry} (UK)
    
    \item competitions
    \begin{itemize}
        \item \href{http://www.kokel.mke.org.hu/}{Kökél} ``Kömal'' for chemistry
        \item \href{http://kemia.apaczai.elte.hu/versenyek/index.htm}{Hungarian chemistry competitions}
        \item \href{https://www.icho.sk/}{IChO} International Chemistry Olympiad
        \item \href{http://www.chem.msu.ru/rus/olimp/welcome.html}{International Mendeleev Chemistry Olympiad} (in Russian), \href{https://www.eko.ut.ee/mko/}{past problems} from 2015 in English 
        \item \href{https://olahverseny.szasz.bme.hu/}{Oláh György Országos Középiskolai Kémiaverseny}
        \item \href{http://curiealapitvany.hu/}{Curie Chemistry Competition}
    \end{itemize}
    \item Textbooks
    \begin{itemize}
        \item \href{https://www.goodreads.com/en/book/show/1957892.Organic_Chemistry}{Organic Chemistry }  by Jonathan Clayden, Nick Greeves, Stuart Warren, Peter Wothers
        \item \href{https://www.goodreads.com/book/show/239513.Principles_of_Biochemistry}{Principles of Biochemistry}  by Albert L. Lehninger, David L. Nelson, Michael M. Cox
    \end{itemize}
    \item Handbooks/Data banks
    \begin{itemize}
        \item \href{https://en.wikipedia.org/wiki/CRC_Handbook_of_Chemistry_and_Physics}{CRC Handbook of Chemistry and Physics}
        \item \href{http://www.ddbst.com/}{Dortmund Data Bank} Thermodynamic data bank (mainly Commercial)
    \end{itemize}
    
    \item fun
    \begin{itemize}
        \item \href{https://www.compoundchem.com/}{info graphics}
        \item \href{https://www.chemistryworld.com/}{Chemistry World} magazine
        \item \href{http://www.periodicvideos.com/}{Periodic Videos} Clips about the Elements
        \item \href{https://www.youtube.com/user/TheRedNile}{NileRed} YT channel
        \item \href{https://fold.it/}{FoldIt} gamification of protein folding problems
    \end{itemize}
    
    \item Other reading lists
    \begin{itemize}
        \item \href{http://www.ox.ac.uk/sites/files/oxford/media_wysiwyg/Introductory_reading_for_Chemistry.pdf}{Oxford} Chemistry
        \item \href{https://www.bioch.ox.ac.uk/recommended-reading-list}{Oxford} Biochemistry
        \item \href{https://www.natsci.tripos.cam.ac.uk/prospective-students/reading}{Cambridge}
    \end{itemize}
    
    \item Internships, projects
    \begin{itemize}
        \item Material science: \href{https://www.bayzoltan.hu/en/home/}{Bay Zoltán Nonprofit Ltd. for Applied Research}
    \end{itemize}
    
\end{itemize}


\subsection{Biology}

\textbf{Disclaimer: I'm not a physician or a biologist, and I didn't do any biology class during my University studies. My view on biology is highly subjective, and probably extremely naive. A contribution of a more experienced biologist/physician would be valuable.}

Biology is of course a huge umbrella term. In my personal view Biology as a field investigates replicators, which can evolve and interact. More specifically Biology is about chemical life, in which DNA/RNA molecules evolve by mutations and complicated interactions. (It is interesting, that this life form is the dominant and most complex one that we know.)

Closely related fields are Physiology and Medicine, which are more anthropomorphic (in my view), and are interested in the functioning and ``health'' of a single individual.

\begin{itemize}
    \item \href{https://www.youtube.com/watch?v=wENhHnJI1ys}{Map of Biology} sketches the main topics in Biology (\href{https://mymodernmet.com/wp/wp-content/uploads/2018/03/infographic-map-biology.png}{poster} version)
    \item \href{https://www.goodreads.com/book/show/145806.Biology}{Campbell Biology} is the main textbook for biology (mainly for undergraduate students). (Online available \href{https://archive.org/details/JaneB.Reece--CampbellBiology9thEd./page/n971/mode/2up}{here})
    \item \href{https://www.goodreads.com/lt/book/show/11991323-the-selfish-gene}{The selfish gene} by  Richard Dawkins is a popular introductory to the genetic (or replicator centered) view of life. (Online available \href{https://archive.org/stream/pdfy-RHEZa8riPwBuUyrV#mode/2up Selfish Gene}{here})
    \item \href{https://www.cambridgeinternational.org/programmes-and-qualifications/cambridge-international-as-and-a-level-biology-9700/}{Cambridge A-level}
    \begin{itemize}
        \item \href{https://www.goodreads.com/book/show/34108108-cambridge-international-as-and-a-level-biology-workbook-with-cd-rom}{Biology A-level Course book} online available \href{https://web.archive.org/web/20201111165810/https://www.gceguide.xyz/files/e-books/a-level/Cambridge\%20International\%20AS\%20and\%20A\%20Level\%20Biology\%20Coursebook.pdf}{here}
    \end{itemize}
    
    \item ``Standard'' online courses
    \begin{itemize}
    \item \href{https://www.khanacademy.org/science/biology}{Biology} on Khan academy
    \item \href{https://www.youtube.com/playlist?list=PL3EED4C1D684D3ADF}{Crash Course Biology}
    \item \href{https://www.youtube.com/playlist?list=PL8dPuuaLjXtNdTKZkV_GiIYXpV9w4WxbX}{Crash Course Ecology}
    \item \href{https://www.youtube.com/playlist?list=PL8dPuuaLjXtOAKed_MxxWBNaPno5h3Zs8}{Crash Course Anatomy \& Physiology}
    
    \item \href{https://ocw.mit.edu/courses/biology/}{Biology} MIT OCW
    \end{itemize}
    
    \item \href{https://study.com/academy/course/introduction-to-biology.html}{Introduction to Biology} on study.com (commercial)
    
    \item \href{https://www.ibiology.org/}{iBiology} a free online resource 
    
    \item \href{https://www.onezoom.org/}{Tree of life} explorer
    
    \item History
    \begin{itemize}
    \item \href{https://www.goodreads.com/book/show/40951.A_History_of_the_Life_Sciences}{A History of the Life Sciences}
    \item \href{https://www.youtube.com/watch?v=llPMfaz4qnA}{The great ideas of biology}
    \end{itemize}
    \item Competitions
    \begin{itemize}
        \item \href{https://www.ibo-info.org/en/}{IBO} International Biology Olympiad
        \item \href{http://biologia.fazekas.hu/kozepiskolai-versenyek/}{Hungarian Biology Competitions} for high school students
    \end{itemize}
    
    \item Other reading lists
    \begin{itemize}
        \item \href{https://biology.stackexchange.com/questions/10376/introductory-biology-text-for-an-outsider}{Introductory Biology texts} from Biology Stack Exchange
        \item \href{https://biology.stackexchange.com/questions/35152/choosing-a-book-to-gain-general-knowledge-about-biology}{List of lists} from Biology Stack Exchange
        \item \href{https://biology.stackexchange.com/questions/21475/books-for-beginners}{Books for beginners} from Biology Stack Exchange
        \item \href{https://www.natsci.tripos.cam.ac.uk/prospective-students/reading}{Cambridge Natural Sciences} reading list
        \item \href{http://www.ox.ac.uk/sites/files/oxford/media_wysiwyg/Introductory_Reading_for_Biomedical_Sciences1.pdf}{Oxford} Biomedical Science
        \item \href{https://www.medsci.ox.ac.uk/study/medicine/pre-clinical/applying/reading}{Oxford} Medical school
        \item \href{https://medicalsciences.stackexchange.com/questions/19310/how-does-one-determine-which-textbook-to-start-with-when-wanting-to-build-knowle/19314}{Medical school introductory advice} from Medical Sciences Stack Exchange 
        \item \href{https://www.medschooladvice.com/best-advice-and-books-for-ms1-ms2-year-classes-and-shelf-exams}{The best books for medical school} Advice \& Resources for the First and Second Years of Medical School
    \end{itemize}
    
    \item Human Biology
    \begin{itemize}
        \item \href{https://www.goodreads.com/book/show/45843.Atlas_of_Human_Anatomy}{Atlas of Human Anatomy}
        \item \href{https://www.zygotebody.com/}{Zygote Body} a 3D anatomy atlas
        \item \href{https://www.visiblebody.com/en-us/}{Visible Body} a 3D anatomy atlas (Commercial)
        \item \href{https://www.youtube.com/playlist?list=PL848F2368C90DDC3D}{Human Behavioral Biology} Lecture series by Robert Sapolsky, on Stanford.
        \item \href{https://www.imdb.com/title/tt0200346/}{The Human Body} BBC educational series
        \item Healthcare statistics
        \begin{itemize}
            \item \href{https://vizhub.healthdata.org/gbd-compare/}{Global Burden of Disease} (GBD)
            \item \href{https://ourworldindata.org/causes-of-death}{Causes of Death} on Our World in Data
            \item \href{https://www.who.int/data/gho}{WHO} data and statistics
        \end{itemize}
    \end{itemize}
    
    \item Mathematical Biology
    \begin{itemize}
        \item \href{https://www.smb.org/}{Society for Mathematical Biology}
        \item \href{https://www.goodreads.com/book/show/13151411-proving-darwin}{Proving Darwin: Making Biology Mathematical} by Gregory Chaitin (The framework and the book is not so great (admitted by Chaitin as well), but the idea might be interesting)
    \end{itemize}
    
    \item fun
    \begin{itemize}
        \item \href{https://www.youtube.com/channel/UCBbnbBWJtwsf0jLGUwX5Q3g}{Journey to the Microcosmos} YT channel
        \item \href{https://www.youtube.com/playlist?list=PLrfcruGtplwH9gqNUy23btvvRuyhGw1sH}{Microscopic Life} YT channel of the American Museum of Natural History
        \item \href{https://www.youtube.com/channel/UCwg6_F2hDHYrqbNSGjmar4w}{Animalogic} YT channel about animals
        \item \href{https://www.spore.com/}{Spore} evolutionary video Game
        \item \href{https://www.minipcr.com/}{Mini PCR} possible home genetic lab
    \end{itemize}
    
    \item \href{https://extendedevolutionarysynthesis.com/}{Extended evolutionary synthesis} is a newer way to look biological evolution \href{https://en.wikipedia.org/wiki/Extended_evolutionary_synthesis}{wiki}.
    
    \item Projects, internships
    \begin{itemize}
        \item \href{https://www.slcu.cam.ac.uk/people/gpsep}{Sainsbury Laboratory in Cambridge}
    \end{itemize}
    
\end{itemize}

\section{Subjects II}

To save further lines of disclaimers I grouped the subjects I know less about together.

A \textbf{General Disclaimer} holds:
\textbf{I'm not a Philosopher / Psychologist / Historian / Economist / Sociologist / Lawyer, etc., and I didn't do any serious formal classes during my University studies regarding these fields. My view on these subjects is highly subjective, and probably extremely naive. A contribution of more experienced scholars/practitioners is much needed.}

Against my limited knowledge in the topics, I tried to collect together useful links and sources I learned from interested students, and fill the gaps with resources I found reliable or useful for my self.
The patchwork of resources, competitions, books, all with varying depth and difficulty can not provide a coherent track for serious self study. However I hope it can help in initial explorations and can enrich a standard curriculum.

\subsection{Philosophy}
\textbf{General Disclaimer: I have limited knowledge in this field.}

\begin{itemize}
    \item \href{https://plato.stanford.edu/}{Stanford Encyclopedia of Philosophy}
    \item \href{https://www.philosophy-olympiad.org/}{International Philosophy Olympiad}
    \item \href{https://www.dialexicon.org/}{Dialexicon} philosophy journal for high school students
    \item \href{https://www.trin.cam.ac.uk/undergraduate/essay-prizes/}{Trinity College Cambridge Essay Competition}
    \item \href{https://bspee.wordpress.com/}{Baltic Sea Philosophy Essay Event}
    \item \href{https://www.amnesty.org.uk/use-your-voice}{``human rights competition'' of Amnesty International}
    \item \href{https://www.immerse.education/essay-competition/}{Cambridge, Oxford \& London Summer School Essay Competition}
    \item book and sources
    \begin{itemize}
        \item \href{https://www.goodreads.com/da/book/show/17563346-lecture-on-ethics}{Lecture on Ethics} by Wittgenstein
        \item \href{https://www.goodreads.com/book/show/332138.After_Virtue}{After Virtue} by  Alasdair MacIntyre
        \item \href{https://cactus.dixie.edu/green/B_Readings/I_Berlin\%20Two\%20Concpets\%20of\%20Liberty.pdf}{Isaiah Berlin, ``TWO CONCEPTS OF LIBERTY''}
    \end{itemize}
\end{itemize}

\subsection{Psychology}
\textbf{General Disclaimer: I have limited knowledge in this field.}


\begin{itemize}
    \item \href{https://www.goodreads.com/book/show/34107392-the-brain}{The Brain} by Gary L. Wenk
    \item \href{https://www.youtube.com/playlist?list=PL6A08EB4EEFF3E91F}{Introduction to Psychology} by Paul Bloom (\href{https://www.coursera.org/learn/introduction-psychology}{Coursera} course)
    \item \href{https://www.edx.org/course/introduction-to-social-psychology}{Introduction to Social Psychology} on edX
   \item books:
   \begin{itemize}
        \item \href{https://www.goodreads.com/book/show/10514920-introduction-to-psychology}{Introduction to Psychology}
       \item \href{https://www.goodreads.com/en/book/show/16242266-disorders-of-childhood}{Disorders of Childhood}
       \item \href{https://www.goodreads.com/book/show/63697.The_Man_Who_Mistook_His_Wife_for_a_Hat_and_Other_Clinical_Tales}{The Man Who Mistook His Wife for a Hat and Other Clinical Tales} by Oliver Sacks
   \end{itemize}
   \item competition
   \begin{itemize}
       \item \href{https://thebrainbee.org/}{Neuroscience Competition for Teens}
   \end{itemize}
    \item Internships, projects
    \begin{itemize}
        \item \href{https://cognitivescience.ceu.edu/unit/baby-lab}{Central European University Baby Lab}
    \end{itemize}
\end{itemize}

\subsection{Medicine}
\textbf{General Disclaimer: I have limited knowledge in this field.}

\begin{itemize}
    \item Books
    \begin{itemize}
        \item \href{https://www.goodreads.com/book/show/7170627-the-emperor-of-all-maladies}{The Emperor of All Maladies}
    \end{itemize}
    \item competitions
    \begin{itemize}
        \item \href{https://thebrainbee.org/}{Brain Bee} Neuroscience Competition for Teens
    \end{itemize}
    \item \href{https://www.thelancet.com/}{The Lancet} peer-reviewed general medical journal
    \item \href{https://www.mayoclinic.org/}{Mayo Clinic}
    \item \href{https://www.nhs.uk/}{NHS} 
    \item magazines
    \begin{itemize}
        \item \href{https://www.healthline.com/}{Healthline}
    \end{itemize}
    \item possible projects:
    \begin{itemize}
        \item \href{https://enablingthefuture.org/}{e-NABLE} makes 3D printed prosthetic upper limb devices
        \item \href{http://koki.hu/english}{Institute of Experimental Medicine} in Hungary
        \item \href{https://www.bethesda.hu/about-us/}{Bethesda} children Hospital
    \end{itemize}
\end{itemize}

\subsection{History/Archeology}
\textbf{General Disclaimer: I have limited knowledge in this field.}


\begin{itemize}
    \item \href{https://mindentudas.videotorium.hu/hu/recordings/8176/mi-a-tortenelem}{What Is History?} by Ignác Romsics in Hungarian (see written version \href{http://real-eod.mtak.hu/1022/1/04\%20Romsics\%20Ign\%C3\%A1c.pdf}{here} and \href{https://mindentudas.hu/el\%C5\%91ad\%C3\%A1sok/tudom\%C3\%A1nyter\%C3\%BCletek/b\%C3\%B6lcs\%C3\%A9szettudom\%C3\%A1ny/138-t\%C3\%B6rt\%C3\%A9nelemtudom\%C3\%A1nyok/6059-a-toertenetiro-dilemmaja-megismerjuek-vagy-csinaljuk-e-a-toertenelmet.html}{here})
    \item \href{https://www.britannica.com/topic/philosophy-of-history}{Philosophy of History}
    \item \href{https://www.youtube.com/channel/UCMFOiJn5wGreCszxZ9xuyTg/playlists}{Hungarian History and Literature} online high school classes in Hungarian
    \item books
    \begin{itemize}
        \item \href{https://www.goodreads.com/book/show/139310.The_Oxford_History_of_Ancient_Egypt}{The Oxford History of Ancient Egypt}
        \item \href{https://www.goodreads.com/book/show/6476537-amarna-sunset}{Amarna Sunset} and \href{https://www.goodreads.com/book/show/18798554-amarna-sunrise}{Amarna Sunrise} by Aidan Dodson
        \href{https://aucpress.com/product/wonderful-things/}{Wonderful Things} by Jason Thompson
        \href{https://www.goodreads.com/en/book/show/355190.Orientalism}{Orientalism} by Edward W. Said
        \href{https://www.cambridge.org/core/books/middle-egyptian/831CD0936A9DC45988F94B74C598353E}{Middle Egyptian} An Introduction to the Language and Culture of Hieroglyphs
        
        \item \href{https://www.goodreads.com/en/book/show/480002.Historical_Dynamics}{Historical Dynamics} or \href{https://www.nature.com/articles/488024a}{Cliodynamics} by Peter Turchin
    \end{itemize}
    \item competition
    \begin{itemize}
        \item \href{https://www.estori.hu/}{Estöri creative history competition}
        \item \href{https://www.st-hughs.ox.ac.uk/prospective-students/outreach-at-st-hughs/the-mary-renault-prize/}{Mary Renault Essay Competition} on a topic relating to the reception of classical antiquity – including Greek and Roman literature, history, political thought, philosophy, and material remains – in any period to the present
        \item \href{https://www.ucl.ac.uk/classics/outreach/essay-competition}{UCL Classics Essay Competition}
    \end{itemize}
\end{itemize}

\subsection{Geography}
\textbf{General Disclaimer: I have limited knowledge in this field.}


\begin{itemize}
    \item books
    \begin{itemize}
        \item \href{https://www.goodreads.com/tr/book/show/41552709-the-uninhabitable-earth}{The Uninhabitable Earth: Life After Warming} by David Wallace-Wells
    \end{itemize}
    \item \href{http://eotvos-tata.edu.hu/versenyek/foldrajz/itthonotthon_index.htm}{``Itthon-Otthon Vagy'' Földrajz Verseny}
    \item Projects, internship
    \begin{itemize}
        \item \href{http://www.interreg-danube.eu/approved-projects/tid-y-up/partners}{Interreg-danube}
    \end{itemize}
\end{itemize}

\subsection{Economics/Politics/Law}
\textbf{General Disclaimer: I have limited knowledge in this field.}


\begin{itemize}
    \item \href{https://www.veryshortintroductions.com/view/10.1093/actrade/9780198726074.001.0001/actrade-9780198726074}{Capitalism: A Very Short Introduction}
    \item \href{https://www.goodreads.com/book/show/46423.Sociology}{Sociology} by Anthony Giddens
    \item \href{https://reconnect-europe.eu/mooc/}{Democracy and the rule of law in the European Union} online course
    \item \href{https://www.coursera.org/learn/security-safety-globalized-world}{Security \& Safety Challenges in a Globalized World} online course
    \item \href{https://www.edx.org/course/justice-2}{Justice} This introduction to moral and political philosophy is one of the most popular courses taught at Harvard College
    (\href{https://www.youtube.com/watch?v=kBdfcR-8hEY}{published} on YouTube as well)
    \item \href{https://www.goodreads.com/en/book/show/10428480-is-eating-people-wrong}{Is Eating People Wrong?}  by Allan C. Hutchinson
    \item projects/internships
    \begin{itemize}
        \item \href{https://21kutatokozpont.hu/index.html}{21 Research Centre} political think tank in Hungary
        \item \href{https://mfcequity.com/}{MFC Equity}
        \item \href{https://www.revas.online/en/}{Revas} online business games
    \end{itemize}
\end{itemize}

\subsection{Literature}
\textbf{General Disclaimer: I have limited knowledge in this field.}


\begin{itemize}
    \item Hungarian literature
    \begin{itemize}
        \item \href{https://mek.oszk.hu/}{Magyar Elektronikus Könyvtár} MEK
        \item \href{https://moly.hu/konyvek/szerb-antal-magyar-irodalomtortenet}{Szerb Antal: Magyar irodalomtörténet} available \href{https://mek.oszk.hu/14800/14871/}{on MEK}
        \item \href{https://moly.hu/konyvek/lackfi-janos-hogyan-irjunk-verset}{Hogyan írjunk verset?} by Lackfi János \href{https://olvassbele.com/2021/02/05/lackfi-janos-hogyan-irjunk-verset-reszlet/}{snippet}
        \item \href{https://moly.hu/konyvek/horvath-viktor-a-vers-ellenforradalma}{A vers ellenforradalma} by Horváth Viktor (on \href{https://www.scribd.com/book/342092705/A-vers-ellenforradalma-A-versiras-es-versforditas-tanulasa-es-tanitasa}{scribd})
        \item \href{https://szepmagyarbeszed.hu/}{Szép Magyar Beszéd} Hungarian Recitation Contest
    \end{itemize}
    \item International Literature
    \begin{itemize}
        \item \href{https://moly.hu/konyvek/szerb-antal-a-vilagirodalom-tortenete}{Szerb Antal: A világirodalom története} available \href{https://mek.oszk.hu/14800/14872/}{on MEK} (in Hungarian)
        \item \href{https://www.babelmatrix.org/}{Babel Matrix}
        \item \href{https://thegreatestbooks.org/}{``The Greatest Books of All Time''}
        \item \href{https://fivebooks.com/category/fiction/world-literature-books/}{World Literature Categories} on \href{https://fivebooks.com/}{Five Books}
        \item \href{https://en.wikipedia.org/wiki/The_Big_Read}{The Big Read} popularity contest for the most loved novels in the UK in 2003 (see Similar contests for more popularity lists) 
        \item \href{https://www.coursera.org/learn/poetry-workshop}{A Poetry Workshop} offered by \href{https://calarts.edu/}{California Institute of the Arts}
        \item \href{https://www.poetryoutloud.org/}{Poetry Out Loud} English (USA) Recitation Contest
        \item \href{https://www.poetryinvoice.com/}{Poetry In Voice} English (Canada) poetry ricitation and writing contest
    \end{itemize}
\end{itemize}

\subsection{Language skills}
\textbf{General Disclaimer: I have limited knowledge in this field.}

\begin{itemize}
    \item English
    \begin{itemize}
        \item \href{https://www.ielts.org/}{IELTS}
        \begin{itemize}
            \item \href{https://www.youtube.com/channel/UCRA4UOCBRgv2w9b59pDMyvQ}{Fastrack IELTS} a YouTube channel by Asiya
            \item \href{https://www.youtube.com/channel/UCQ3A7Dnyz1_Fxaa5BCzAPMA}{AcademicEnglishHelp} another YT channel for Academic IELTS
        \end{itemize}
        \item \href{https://www.ets.org/toefl}{TOEFL}
        \item \href{https://learnenglish.britishcouncil.org/}{British Council}
        \item Dictionaries, Synonyms
        \begin{itemize}
            \item \href{https://translate.google.com/}{Google Translate}
            \item \href{https://www.thesaurus.com/}{Thesaurus}
            \item \href{https://www.merriam-webster.com/}{Merriam Webster}
            \item \href{https://www.deepl.com/translator}{DeepL Translator}
        \end{itemize}
        \item Style and Grammar
        \begin{itemize}
            \item \href{https://www.britishcouncil.org/voices-magazine/what-are-correct-rules-english-grammar}{General post} about the non-prescriptive nature of English grammar
            \item \href{https://learnenglish.britishcouncil.org/grammar/english-grammar-reference}{English Grammar Reference} from the British Council
            \item \href{https://www.goodreads.com/book/show/75441.The_New_York_Times_Manual_of_Style_and_Usage}{The New York Times Manual of Style and Usage}
            \item \href{https://www.apstylebook.com/}{AP Stylebook}
            \item \href{https://en.wikipedia.org/wiki/Style_guide}{Other Style guides}
        \end{itemize}
    \end{itemize}
    \item German
    \begin{itemize}
        \item \href{https://www.goethe.de/en/index.html}{Goethe-Institut}
    \end{itemize}
    \item French
    \begin{itemize}
        \item \href{http://www.delfdalf.fr/index-en.html}{DELF DALF}
    \end{itemize}
    \item Spanish
    \begin{itemize}
        \item \href{https://www.dele.org/}{DELE}
    \end{itemize}
    \item Slovak
    \begin{itemize}
        \item \href{https://www.e-slovak.sk/}{e-slovak}
        \item \href{https://www.cseregyerek.sk/}{Cseregyerek} exchange program for 8-15 years old Hungarian children living in Slovakia 
    \end{itemize}
\end{itemize}

\section{Complementary materials}

\subsection{YouTube channels/Podcasts}

\begin{itemize}
    \item YouTube channels
    \begin{itemize}
    \item \href{https://www.youtube.com/user/TheRoyalInstitution}{The Royal Institution}
    \item \href{https://www.youtube.com/user/Kurzgesagt}{Kurzgesagt}
    \item \href{https://www.youtube.com/user/sixtysymbols}{Sixty Symbols}
    \item \href{https://www.youtube.com/channel/UCYO_jab_esuFRV4b17AJtAw}{3Blue1Brown}
    \item \href{https://www.youtube.com/channel/UCxqAWLTk1CmBvZFPzeZMd9A}{Domain of Science}
    \item \href{https://www.youtube.com/channel/UCoxcjq-8xIDTYp3uz647V5A}{Numberphile}
    \item \href{https://www.youtube.com/channel/UC1_uAIS3r8Vu6JjXWvastJg}{Mathologer}
    \item \href{https://www.youtube.com/channel/UCtESv1e7ntJaLJYKIO1FoYw}{Periodic Videos}
    \item \href{https://www.youtube.com/channel/UC6107grRI4m0o2-emgoDnAA}{SmarterEveryDay}
    \item \href{https://www.youtube.com/channel/UC0cd_-e49hZpWLH3UIwoWRA}{Professor Dave Explains}
    \item \href{https://www.youtube.com/channel/UC2C_jShtL725hvbm1arSV9w}{CGP Grey}
    \item \href{https://www.youtube.com/user/1veritasium}{Veritasium}
    \item \href{https://www.youtube.com/channel/UC6nSFpj9HTCZ5t-N3Rm3-HA}{Vsauce}
    \item \href{https://www.youtube.com/user/minutephysics}{minutephysics}
    \item \href{https://www.youtube.com/minuteearth}{minuteearth}
    \item \href{https://www.youtube.com/channel/UC7DdEm33SyaTDtWYGO2CwdA}{Physics Girl}
    \item \href{https://www.youtube.com/channel/UCSIvk78tK2TiviLQn4fSHaw}{Up and Atom}
    \item \href{https://www.youtube.com/playlist?list=PLOYRlicwLG3St5aEm02ncj-sPDJwmojIS}{Map of Science}
    \item \href{https://www.youtube.com/channel/UC9-y-6csu5WGm29I7JiwpnA}{Computerphile}
    \item \href{https://www.youtube.com/channel/UC7IcJI8PUf5Z3zKxnZvTBog}{The School of Life}
    \item \href{https://www.youtube.com/user/gradyhillhouse}{Practical Engineering}
    \item \href{https://www.youtube.com/user/GeographyNow}{Geography Now}
    \item \href{https://www.youtube.com/channel/UCNhX3WQEkraW3VHPyup8jkQ}{Langfocus}
    \item \href{https://www.youtube.com/channel/UCKY00CSQo1MoC27bdGd-w_g}{Crash course for Aliens} from Zogg
    \item \href{https://www.youtube.com/channel/UCKMnl27hDMKvch--noWe5CA}{Cogito}
    \item \href{https://www.youtube.com/channel/UC9dRb4fbJQIbQ3KHJZF_z0g}{Let's Talk Religion}
    \item \href{https://www.youtube.com/channel/UCHsRtomD4twRf5WVHHk-cMw}{TierZoo}
    \item \href{https://www.youtube.com/channel/UCJjSDX-jUChzOEyok9XYRJQ}{Jubelee}
    \item \href{https://www.youtube.com/channel/UCgRBRE1DUP2w7HTH9j_L4OQ}{Medlife Crisis}
    \item \href{https://www.youtube.com/channel/UCY7dD6waquGnKTZSumPMTlQ}{Oxford Union}
    \item \href{https://www.youtube.com/channel/UCenxjWEkb0Sv67vejOgZ3Tg}{IntelligenceSquared Debates}
    \item \href{https://www.youtube.com/c/talksatgoogle}{Talks at Google}
    \end{itemize}
    
    \item Other podcasts
    \begin{itemize}
    \item \href{https://www.npr.org/podcasts/510308/hidden-brain}{Hidden Brain}
    \item \href{https://www.thenakedscientists.com/}{The Naked Scientists}
    \item \href{https://ed.ted.com/}{TED Ed}
    \item \href{https://www.preposterousuniverse.com/podcast/}{Sean Carroll's Mindscape Podcast}
    \item \href{https://freakonomics.com/}{Freakonomics}
    \item \href{https://bloggingheads.tv/}{Blogging heads} (USA politics, news, etc.)
    \end{itemize}
    
    \item fun
    \begin{itemize}
        \item \href{http://htwins.net/scale2/}{The Scale of the Universe} other examples from \href{https://en.wikipedia.org/wiki/Orders_of_magnitude_(length)}{Wikipedia}
    \end{itemize}
    
    \item Hungarian
    \begin{itemize}
        \item \href{https://mindentudas.hu/}{Mindentudás Egyeteme} further videos \href{https://mindentudas.videotorium.hu/hu/channels/1132/mindentudas-egyeteme}{here}.
        \item \href{http://www.atomcsill.elte.hu/}{Atomcsill}, Az atomoktól a csillagokig
        \item \href{http://szertar.com/}{Szertár}
        \item \href{https://www.virusklub.hu/}{Vírus Klub}
        \item \href{https://mediaklikk.hu/musor/mindenki-akademiaja/}{Mindenki Akadémiája}
    \end{itemize}

\item Other lists
\begin{itemize}
    \item \href{https://blog.feedspot.com/educational_youtube_channels/}{Top 100 Educational YouTube Channels}
    \item \href{https://medium.com/the-graph/60-youtube-channels-that-will-make-you-smarter-44d8315c2548}{60 YouTube channels}
    \item \href{https://collegeinfogeek.com/educational-youtube-channels/}{100+ YT channels}
\end{itemize}

\end{itemize}

\subsection{Books, reading lists}


\begin{itemize}

    \item \href{https://www.goodreads.com/book/show/5129.Brave_New_World}{Brave New World} by Aldous Huxley
    \item \href{https://www.goodreads.com/book/show/242472.The_Black_Swan}{The Black Swan: The Impact of the Highly Improbable} by by Nassim Nicholas Taleb 
    \item \href{https://www.goodreads.com/book/show/34890015-factfulness}{Factfulness} by Hans Rosling
    \item \href{https://www.goodreads.com/book/show/23692271-sapiens}{Sapiens} by Yuval Noah Harari
    \item \href{https://www.goodreads.com/book/show/34272565-life-3-0}{Life 3.0: Being Human in the Age of Artificial Intelligence} by Max Tegmark
    \item Reading lists
    \begin{itemize}
        \item \href{https://bookriot.com/bill-gates-book-recommendations/}{Bill Gates collected recommendations} from 2012 to 2020
        \item \href{https://www.goodreads.com/series/152497}{The Big Questions Series}
    \end{itemize}
\end{itemize}

\subsection{Real life events}

\subsubsection{Science festivals}

\begin{itemize}
    \item \href{https://kutatokejszakaja.hu/}{Kutatók északája} (in Hungary)
    \item \href{https://pintofscience.com/}{Pint of Science}
    \item \href{https://opendays.cern/}{CERN open days}
\end{itemize}

\subsubsection{Science museums}

\begin{itemize}
    \item \href{https://www.csopa.hu/}{Csodák Palotája}
    \item \href{https://momath.org/}{Museum of Mathematics}
    \item \href{https://www.sciencemuseum.org.uk/home}{Science Museum London}
    \item \href{https://www.mos.org/}{Museum of Science Boston}
    
    \item \href{http://semmelweismuseum.hu/}{Semmelweis medical museum}
\end{itemize}

\subsection{Summer schools}

\begin{itemize}
    \item \href{http://drustvo-evo.hr/s3/}{$S^3/S^3{\mathrm{++}}$} in Croatia
    \item \href{https://education.wolfram.com/summer-camp/}{Wolfram High School Summer Camp} in Boston
    \item \href{https://en.agondolkodasorome.hu/summercamps/#}{MaMuT summer camp} in Hungary
    \item \href{https://mathsbeyondlimits.eu/}{Maths Beyond Limits} in Poland
    \item \href{https://www.insead.edu/summer-at-insead}{INSEAD} Institut Européen d'Administration des Affaires in France
    \item \href{https://www.immerse.education/}{Immerse Education} in England
    \item \href{https://www.varazslatos-kemia-tabor.mke.org.hu/}{Varázslatos kémia tábor} in Hungary
\end{itemize}

\subsubsection{Other competitions}

\begin{itemize}
    \item \href{https://www.oxfordschools.net/}{Oxford Schools} debating competition
    \item \href{https://milestone-institute.org/student-life/debate/}{National Debate Qualifiers}
    \item \href{https://wsdcdebate.org/about-wsdc}{World Schools Debating Championship}
    \item \href{https://www.jugend-debattiert.eu/}{Jugend Debattiert}
    \item \href{https://www.johnlockeinstitute.com/essay-competition}{John Locke Essay Competition}
    \item \href{https://www.kutdiak.hu/en/events/}{National Conference of Researching Students} (Kutdiák)
    \item \href{https://www.kidfilmfestival.hu/en/}{Cinemira} International Children's Film Festival
    \item \href{https://technovationchallenge.org/}{Technovation} innovation based team competition for girls
    \item \href{https://lim.lauder.hu/}{24h Innovation Marathon} by Lauder Javne School
    \item \href{https://www.innovacio.hu/en_3a.htm}{National Scientific and Innovation Contest for Youth}
    \item \href{https://hungary.socialimpactaward.net/en/}{Social Impact Award}
    \item \href{https://ecolymp.org/}{International Economics Olympiad}
    \item \href{http://megapolis.educom.ru/en}{International Olympiad of Metropolises}
    \item \href{https://www.jugend-forscht.de/}{Jugend Forscht}
    \item \href{https://businessismore.eu/}{High School Business Challenge}
    \item \href{https://www.firstlegoleague.org/}{First Lego League competition}
    \item \href{https://en.wikipedia.org/wiki/Model_United_Nations}{Model United Nations} find a list of events \href{https://mymun.com/}{here} and see \href{https://www.bimun.hu/}{BIMUN}, \href{http://munapest.com/}{Munapest} and \href{https://www.mun.bme.hu/}{BME MUN} for events in Budapest
    \href{https://mepeurope.eu/}{Model European Parliament}
    \item \href{https://tanulmanyiversenyek.hu/}{Collection of national competitions} in Hungary
    \item \href{https://www.nytimes.com/spotlight/learning-contests}{The New York Times Contest} including the \href{https://www.nytimes.com/2021/04/16/learning/our-12th-annual-summer-reading-contest.html}{New York Times Summer Reading Contest}
    \item \href{http://youth.worldbridge.org/}{Competitive Bridge} card game
    \item \href{https://worldskills.org/}{World Skills}
\end{itemize}

\subsection{Teaching/Educational organizations}

\begin{itemize}
    \item \href{https://milestone-institute.org/}{Milestone Institute}
    \item \href{https://www.euroexam.org/go2uni}{go2uni}
    \item \href{https://romaversitas.hu/en/}{Romaversitas}
    \item \href{https://www.kutdiak.hu/en/}{Kutdiak}
    \item \href{https://www.crimsonglobalacademy.school/uk/}{Crimson Global Academy}
    \item \href{https://www.etoncollege.com/}{Eton College}
    \item \href{https://summerinstitutes.spcs.stanford.edu/}{Pre-Collegiate Summer Course at Stanford University}
    \item \href{https://globalscholars.yale.edu/}{Yale Young Global Scholars program}
    \item \href{https://www.nyas.org/programs/global-stem-alliance/1000-girls-1000-futures/}{1000 Girls 1000 Futures}
    \item \href{https://nokatud.hu/smartiz/}{Smartiz} multidisciplinary STEM program for girls
    \item \href{https://www.akg.hu/}{Alternatív Közgazdasági Gimnázium}
    \item \href{https://www.mindsunderground.com/}{Minds Underground}
    \item \href{https://heterodoxacademy.org/}{Heterodox Academy}
    
    \item Scholarships
    \begin{itemize}
        \item \href{https://www.assistscholars.org/en/index}{ASSIST}
    \end{itemize}
\end{itemize}

\subsection{Volunteering/Activism}

Reflecting critically on our social circumstance and being an active members of society is a virtue in our modern world.
However, one needs knowledge and wisdom to see the relevant problems, and determination to act effectively.
While learning your civic roles,
be critical, be active, and be brave to revise your own goals and actions time to time.




\subsubsection{Real world challenges}
Real world challenges does not need to be global. However there are some, which are widely recognised:
\begin{itemize}
    \item \href{https://reports.weforum.org/global-risks-report-2020/}{Global Risk Report} by the World Economic Forum
    \item \href{https://www.ipcc.ch/}{The Intergovernmental Panel on Climate Change}
\end{itemize}

\subsubsection{Organisations}
There are many organizations, where one can do volunteering. Some of these are international, some national, and some focusing on a small area.
But keep in mind, that you can be active even without joining an organization (for instance by picking up trash in nature).
\begin{itemize}
    \item \href{https://afs.org/}{AFS}
    \item \href{https://helsinki.hu/en/support/apply-to-volunteer/}{Hungarian Helsinki Committee}
    \item \href{https://hclu.hu/en/about-us}{Hungarian Civil Liberties Union}
    \item \href{https://unicef.hu/igy-segithetsz/onkentes}{UNICEF Hungary}
    \item \href{https://www.unv.org/}{UN}
    \item \href{https://www.politicalcapital.hu/}{Political Capital} Hungarian Political think-tank
    \item \href{https://adommozgalom.hu/}{ADOM mozgalom} Hungarian high school youth movement
    \item \href{https://www.ecolinst.hu/index.php}{Ökológiai Intézet}
    \item \href{https://web.archive.org/web/20171124170411/http://foncsorozo.hu/napsukar/}{Foncsorozó, NapSukár} on \href{https://www.facebook.com/napsukar/}{Facebook}
    \item \href{https://mas-zinhaz.hu/}{MáSzínház}
    \item \href{https://hintalovon.hu/en/child-participation-policy/}{Hintalovon Foundation}
    \item \href{https://korhazsuli.hu/onkenteseknek/}{KórházSuli}
    \item \href{http://www.etanoda.hu/segits_oktatassal_gyere_hozzank_mentornak_207}{etanoda}
    \item \href{https://fridaysforfuture.org/}{Fridays for Future}
\end{itemize}

\section{University choice}

In the 21st century any theoretical material can be learned if someone has a working internet connection. There is a huge variety of online courses, many books and papers are accessible. However, Universities are still useful for:

\begin{itemize}
    \item Providing practical classes and access to laboratories
    \item Socializing with your classmates, which can provide a valuable professional network
    \item Changing and shaping your world view
    \item Can give opportunities for research (mainly during MSc and PhD)
    \item Accommodates you to the academic workload
\end{itemize}

There are three main University ranking sites, which provide a lot of additional useful information.
\begin{itemize}
\item \href{https://www.timeshighereducation.com/world-university-rankings/2020/world-ranking}{Times Higher Education}
\item \href{https://www.topuniversities.com/university-rankings/world-university-rankings/2020}{QS}
\item \href{http://www.shanghairanking.com/}{Shanghai Ranking}
\end{itemize}

\subsection{Interview tips}

There are some universities, when an interview is part of the application process. Before preparing for this round, I think the most important thing is to realize what the interview process is for.

Here the interviewers are usually {\bf not} interested in how flawlessly the candidate can answer all the questions, but in the following 3 main aspects:

\begin{itemize}
    \item Interest
    \item Grasp of discipline-appropriate way of thinking
    \item Teachability
\end{itemize}

These are mainly meta-learning skills, for which practicing previous interview questions does not help. To figure out the mentioned aspects, the interviewers will ask hard questions, to the point when the candidate needs to figure something out, provide reasons, and take hints / help from the interviewers. If one can calmly form a coherent (but not necessarily perfect) argument and incorporate hints, then not knowing an answer here is not a bug but a feature.

Be prepared to talk about your interests, your motivation to apply to the specific place, and be open for problem solving and reasoning.

For more information see:

\begin{itemize}
    \item \href{https://www.ox.ac.uk/admissions/undergraduate/applying-to-oxford/guide/interviews}{Oxford} interview
    \item \href{https://www.undergraduate.study.cam.ac.uk/applying/interviews}{Cambridge} interview
\end{itemize}

Past interview questions are available \href{https://ianramsey.org.uk/wp-content/uploads/2020/08/Sample-Oxbridge-Interview-Questions.pdf}{here} and \href{https://www.thatoxfordgirl.com/post/50-real-life-oxford-university-interview-questions}{here}. (There are some other sources, including a  \href{https://www.amazon.com/Ultimate-Oxbridge-Interview-Guide-UniAdmissions/dp/0993231136}{book}, but I think it is not essential for preparation.)

For online interviews one can get familiar with online shareable drawing tools like \href{https://miro.com/app/dashboard/}{Miro}.

\section{Elements of Pastoral care}

In this section I want to share my highly subjective thoughts about psychological difficulties, existential crisis, burn out, well being, finding identity and finding meaning in life.
Naturally, I will not provide solid answers to these questions, but I want to summarize the sources and directions which I found useful for myself, for my close friends and relatives, and my former mentees.

In this section, I will use quotes and poetry, not because they are good in transmitting information precisely, but because they can serve as mirrors, and one can explore and understand themselves by them. Because this section is mostly about the reader, yes, about you.

\vspace{1cm}
{``Believe Those Who Are Seeking the Truth; Doubt Those Who Find It''
\\[5pt]
\rightline{{\rm --- \href{https://quoteinvestigator.com/2013/11/14/seekers/}{Doesn't Really Matter}}}
}

\subsection{If you need help}

There are situations, which are clearly harmful, and intervention is needed. If you face abuse, or you are in danger otherwise, don't hasitate to reach out for consultation and help:

\begin{itemize}
\item Hungarian
\begin{itemize}
    \item \href{https://kek-vonal.hu/}{Kék vonal} 116 111 (they do pick up the phone, and you can have an anonymous conversation)
    \item \href{https://sos116-123.hu/}{Magyar Lelki Elsősegély Telefonszolgálatok
Szövetsége (LESZ)} 116 123
    \item \href{https://nane.hu/}{NANE} +36 80 505 101
    \item \href{https://tasz.hu/ingyenes-jogsegelyszolgalat}{TASZ ingyenes jogsegély} +36 1 279 22 35
    \item \href{https://hintalovon.hu/}{Hintalovon}
    \item \href{https://yelon.hu/}{Yelon}
\end{itemize}

\item International
\begin{itemize}
\item \href{https://www.childhelplineinternational.org/child-helplines/child-helpline-network/}{Child Helpline Networks}
\end{itemize}
\end{itemize}

\subsection{Changing your self and/or changing the World}

\vspace{1cm}
{``God, grant me the serenity to accept the things I cannot change,\\
courage to change the things I can,\\
and wisdom to know the difference.''
\\[5pt]
\rightline{{\rm --- \href{https://en.wikipedia.org/wiki/Serenity_Prayer}{Serenity Prayer}}}
}

I assume, that the reader of these lines is a teenager or a young adult. (And also I assume, that the reader is not in a miserable situation, where only by apathy or strong faith can they survive the days. I assume that they are not in luxury either from where injustice is merely visible.)

More often than not, people in their young ages observe injustice, unfairness, hypocrisy, etc. in the World, and they are annoyed by it and some of their suffering comes from the difference between how the World ``should be'' and how it is. This is perfectly normal, and absolutely necessary in a \href{https://en.wikipedia.org/wiki/Modernity}{modern society}, where change is in many cases seen as constructive and not as a destructive force. 

To suggest some material about change and the possible ways to it I suggest Barack Obama's relatively sober \href{https://www.youtube.com/watch?v=Ioz96L5xASk}{advice} on the matter.


\subsection{Well being}

Well being is a highly subjective, and not easy to \href{https://plato.stanford.edu/entries/well-being/}{define} concept.
However, to give some simple and hopefully useful advice, I would list a few relevant concepts popularized by \href{https://en.wikipedia.org/wiki/Amit_Sood}{Amit Sood}:

\begin{itemize}
    \item Gratitude
    \item Compassion
    \item Acceptance
    \item Meaning
    \item Forgiveness
\end{itemize}

A more detailed list, containing elements, which in my view can be relevant when one faces difficulties, or wants to be more resilient or balanced:

(This is not a check list, and there is no objective grading between the concepts. It can happen, that for you some elements are crucial, while others are irrelevant. These are only aspects, which often can come up as relevant factors.)

\begin{itemize}
    \item Meaning
    \begin{itemize}
        \item Meaning is both a deep concept, and a meaningless cliché.
        \item \href{https://greatergood.berkeley.edu/article/item/happy_life_different_from_meaningful_life}{Meaning and Happiness}
        \item \href{https://www.brainpickings.org/2013/03/26/viktor-frankl-mans-search-for-meaning/}{Viktor Frankl} 
        \item \href{https://en.wikipedia.org/wiki/Ikigai}{Ikigai}
    \end{itemize}
    \item Food
    \begin{itemize}
        \item Food and diet can lead to heated debate even between experts. Because of that my suggestion is not to search for the ``best'' diet, but to be a bit more conscious, make incremental changes, and figure out which diet would fit best into our own life.
        \item The link between mental health and diet is summarized \href{https://www.mentalhealth.org.uk/a-to-z/d/diet-and-mental-health}{here} and on this \href{https://www.bbc.co.uk/food/articles/diet_wellbeing}{BBC article}
        \item a very basic introduction on BBC \href{https://www.bbc.co.uk/bitesize/topics/zf339j6/articles/zmwvgdm}{Bitsize}
        \item food has a big impact on \href{https://www.healthline.com/nutrition/healthy-eating-for-beginners}{health and quality of life}
        \item \href{https://www.annualreviews.org/doi/full/10.1146/annurev-publhealth-032013-182351}{Diet comparing study}
    \end{itemize}
    \item Physical activity
    \begin{itemize}
        \item The link between mental health and physical exercise is summarized \href{https://www.mentalhealth.org.uk/publications/how-to-using-exercise}{here}.
        \item here are some more \href{https://www.mayoclinic.org/diseases-conditions/depression/in-depth/depression-and-exercise/art-20046495}{tips}.
    \end{itemize}
    \item Sleep
    \begin{itemize}
        \item link between \href{https://www.healthline.com/nutrition/10-reasons-why-good-sleep-is-important#10.-Sleep-affects-emotions-and-social-interactions}{sleep and heath}
        \item a few \href{https://www.healthline.com/nutrition/ways-to-fall-asleep#_noHeaderPrefixedContent}{tips}
        \item and a few \href{https://mindfulnessexercises.com/sleep-meditation-scripts/}{mindfulness tips}
    \end{itemize}
    \item Compassion
    \begin{itemize}
        \item ``feeling for another''
        \item a somewhat relevant \href{https://www.goodreads.com/quotes/6697537-everyone-you-meet-is-fighting-a-battle-you-know-nothing}{quote}, which has many \href{https://quoteinvestigator.com/2010/06/29/be-kind/#more-778}{variants} : ``Everyone you meet is fighting a battle you know nothing about. Be kind. Always.''
        \item compassion and empathy is not only a deeply rooted instinct and possibly foundation of many aspects of morality and ethics, but by cultivating it, one can expand even its identity beyond the spacial and temporal boundaries we usually think our self is confined in.
    \end{itemize}
    \item Social network
    \begin{itemize}
    \item Long lasting, meaningful social relationships, with friends and family was found as the most important factor in \href{https://news.harvard.edu/gazette/story/2017/04/over-nearly-80-years-harvard-study-has-been-showing-how-to-live-a-healthy-and-happy-life/}{Harvard Study of Adult Development}.
    Keep in mind, that this is a precious resource, and put energy to develop and maintain your connections, prioritizing quality over quantity.
    \end{itemize}
    \item Romantic relationship
    \begin{itemize}
        \item Romantic relationships, seduction, long term relationships, marriage and sexuality is widely debated topic on different levels, and from different aspects. This note can not aim to give a complete discussion of the topic, it tries only to give some starting points and maybe two suggestions:
        Romantic relationships are changing, because the world around us is changing, this is why creativity and flexibility starts to be more and more important in long term relationships, and why old customs don't always work.
        There is no one single method for seduction, and real romantic relationships can not be measured in success rates. That is why I would warn you against the so called \href{https://en.wikipedia.org/wiki/Pickup_artist}{Pickup artist} movement, which targets mainly young heterosexual males, and beside teaching some basic psychology and manipulative techniques totally misses the point by basically objectifying women.
        \item \href{https://www.nytimes.com/2015/01/09/style/no-37-big-wedding-or-small.html}{There are 36 questions} which can help to start meaningful conversations
        \item \href{https://www.healthline.com/health/being-in-love}{Loving and being in Love}
        \item \href{https://www.youtube.com/watch?v=3E46oWB4V0s}{The psychology of seduction}
        \item \href{https://www.bbc.com/reel/video/p07l3r3q/unpacking-the-psychology-of-seduction}{BBC clip}
        \item \href{https://en.wikipedia.org/wiki/Esther_Perel}{Esther Perel} an introductory \href{https://www.ted.com/talks/esther_perel_the_secret_to_desire_in_a_long_term_relationship}{TED talk} and her book \href{https://www.goodreads.com/book/show/27485.Mating_in_Captivity}{Mating in Captivity}
        \item \href{https://www.youtube.com/user/sexplanations/playlists}{sexplanations}
        \item \href{https://www.ncbi.nlm.nih.gov/pmc/articles/PMC6650954/}{study}
    \end{itemize}
    \item Natural environment
    \begin{itemize}
        \item \href{http://www.bbc.com/earth/story/20160420-how-nature-is-good-for-our-health-and-happiness}{BBC article}
        \item \href{https://en.wikipedia.org/wiki/NASA_Clean_Air_Study}{NASA Clean Air Study}
    \end{itemize}
    \item Reasonable comfort
    \begin{itemize}
        \item in some cases there are scholarship opportunities even for \href{https://eduline.hu/kozoktatas/Kozepiskolaskent_is_kaphatsz_osztondijat_VZXL0D}{high school students}
    \end{itemize}
\end{itemize}

\subsubsection{Meaning}

\vspace{1cm}
{``He who has a why to live for can bear with almost any how.''
\\[5pt]
\rightline{{\rm \rotatebox[origin=c]{180}{--- Friedrich Nietzsche}}}
}

Meaning is not something we only find, it is something we create. It is maybe not objectively out there, but most people need to feel that they and/or their life has a meaning.

\vspace{1cm}
{``For success, like happiness, cannot be pursued; it must ensue, and it only does so as the unintended side-effect of one’s personal dedication to a cause greater than oneself or as the by-product of one’s surrender to a person other than oneself.''
\\[5pt]
\rightline{{\rm \rotatebox[origin=c]{180}{--- Viktor Frankl}}}
}


\vspace{1cm}
{``Drive overrides fear''
\\[5pt]
\rightline{{\rm \rotatebox[origin=c]{180}{--- Elon Musk}}}
}


\subsection{Practices}

\begin{itemize}
    \item Meditation. (But be careful! \href{https://www.youtube.com/watch?v=P8PCsy268As}{Matthieu Ricard})
    \begin{itemize}
    \item doing a meditation is profoundly easy, and in its basic form it is nothing else then getting familiar with your inner workings by patient and accepting observation. For example you sit down for 10-15 minutes, and just observe your mind and body without any specific aim.
    \item single sessions of meditation will probably not make any difference, but a daily cultivation can gradually have an effect on your mind and even on your body and brain. Similarly how teeth brushing works.
    \item for more detailed basic instructions see this \href{https://www.mindful.org/how-to-meditate/}{guide}
    \item a \href{https://www.sciencefocus.com/news/can-mindfulness-and-meditation-be-harmful/}{word of caution}: meditation (or getting more conscientious about your body and mind) is not a feelgood exercise only, and if you have unresolved conflicts, they can come up, and if you have rigid believes about reality and your self they can be altered. Go gradually, and ask for help, if you feel too uneasy in a situation.
    \end{itemize}
    \item Cold exposure
    \begin{itemize}
        \item also known as the \href{https://www.wimhofmethod.com/}{Wim Hof method}. See a \href{https://www.youtube.com/watch?v=D6EPuUdIC1E}{summary} about the matter
        \end{itemize}
    \item Yoga
    \begin{itemize}
        \item \href{https://www.facebook.com/kosazoe}{ZoeYoga} Beginner - intermediate (Hungarian)
        \item \href{https://www.youtube.com/channel/UCmHii9X_Ct0uWGOE-DibvbQ}{Leigha Butler} all levels
        \item \href{https://www.goodreads.com/book/show/56301.Light_on_Yoga}{Light on Yoga} by B.K.S. Iyengar
    \end{itemize}
\end{itemize}


\subsection{Meaning of life}

\vspace{1cm}
{``Life is like music for its own sake. We are living in an eternal now, and when we listen to music we are not listening to the past, we are not listening to the future, we are listening to an expanded present.''
\\[5pt]
\rightline{{\rm \rotatebox[origin=c]{180}{--- Alan Watts \href{https://www.youtube.com/watch?v=rBpaUICxEhk}{Life as Music}}}}
}

\vspace{1cm}
{``Follow your bliss''
\\[5pt]
\rightline{{\rm \rotatebox[origin=c]{180}{--- Joseph Campbell}}}
}

\vspace{1cm}
{``We are the cosmos made conscious and life is the means by which the universe understands itself.''
\\[5pt]
\rightline{{\rm \rotatebox[origin=c]{180}{--- Brian Cox (and many others including \href{https://www.youtube.com/watch?v=RuXCAtWMFCA}{Alan Watts})}}}
}

\vspace{1cm}
{``I've told thee, man, strive and trust! ''
\\[5pt]
\rightline{{\rm \rotatebox[origin=c]{180}{--- Imre Madách, \href{https://archive.org/details/tragedyofmandram00madrich}{The tragedy of man}}}}
}

The philosophy of Kurzgesagt: \href{https://www.youtube.com/watch?v=MBRqu0YOH14}{Optimistic Nihilism}

Friedrich Nietzsche
\href{https://medium.com/the-sophist/nietzsches-three-steps-to-a-meaningful-life-f063793adfc4}{The story of the camel, the lion, and the child}

\subsection{Rules, Paths, Advice}

There is no ultimate rule book for life. Every list will only grasp a little fraction of the complexity of existence. To show different flavours and the inconsistent nature of rules I listed some popular / interesting ones together.



\subsubsection{The Noble Eightfold Path}

\begin{enumerate}
    \item Right understanding (Samma ditthi)
    \item Right thought (Samma sankappa)
    \item Right speech (Samma vaca) 
    \item Right action (Samma kammanta)
    \item Right livelihood (Samma ajiva) 
    \item Right effort (Samma vayama)
    \item Right mindfulness (Samma sati)
    \item Right concentration (Samma samadhi)
\end{enumerate}
The Buddhist tradition is 2500 years old, and counting. Because of that I believe it needs some context. I suggest \href{https://www.youtube.com/watch?v=Hhlj_SU9SAE}{Secular Buddhism} lecture by Stephen Batchelor.
(Right understanding traditionally refers to accepting reinkarnation. However, the process seems to work ``only'' by accepting the rule of cause and effect, and the non absolute status of our own ego.)

To have a little broader historical view on Buddha and Buddhism I suggest this \href{https://www.youtube.com/watch?v=ulSlL3ubJ3c}{documentary}.

\iffalse
\begin{minipage}[t]{0.45\textwidth}

...

\end{minipage}
\hfill
\begin{minipage}[t]{0.45\textwidth}

...

\end{minipage}
\fi


\subsubsection{Discipline}

\begin{minipage}[t]{0.45\textwidth}

\begin{enumerate}
    \item Stand up straight with your shoulders back
    \item Treat yourself like you are someone you are responsible for helping
    \item Make friends with people who want the best for you
    \item Compare yourself with who you were yesterday, not with who someone else is today
    \item Do not let your children do anything that makes you dislike them
    \item Set your house in perfect order before you criticize the world
    \item Pursue what is meaningful (not what is expedient)
    \item Tell the truth — or, at least, don’t lie
    \item Assume that the person you are listening to might know something you don’t
    \item Be precise in your speech
    \item Do not bother children when they are skate-boarding
    \item Pet a cat when you encounter one on the street
\end{enumerate}
Jordan B Peterson
\href{https://www.goodreads.com/book/show/30257963-12-rules-for-life}{12 Rules for Life: An Antidote to Chaos}

\end{minipage}
\hfill
\begin{minipage}[t]{0.45\textwidth}

\begin{enumerate}
    \item Make your bed
    \item Find people to paddle with you
    \item Measure the size of heart, not flippers
    \item Get over being a sugar cookie and keep moving forward
    \item Don't be afraid of the circuses
    \item Sometimes you have to slide down obstacles head first
    \item Don't back down from the sharks
    \item You must be your very best in the darkest moments
    \item Start singing when you're up to your neck in mud.  Hope for everyone
    \item Don't ever, ever ring the bell
\end{enumerate}
Admiral William H. McRaven

See his \href{https://www.youtube.com/watch?v=pxBQLFLei70}{Commencement Address} which is a summary of his \href{https://www.goodreads.com/book/show/31423133-make-your-bed}{book}.

\end{minipage}

\vspace{12pt}

Jordan Peterson is a polarising character, however he and his message become very popular in a short amount of time, possibly reflecting a hunger for similar father figures. Personally I think discipline and strength are not the single most important skills we need in life, but I mention this school, because some people in some stages of their life might need some push in these areas, and probably they can resonate to this kind of messages.\footnote{However I would mention, that a similar kind of Christianity inspired self-hep book is not entirely new. Here I would mention \href{https://en.wikipedia.org/wiki/M._Scott_Peck}{M. Scott Peck} and his book \href{https://www.goodreads.com/book/show/347852.The_Road_Less_Traveled}{The Road Less Traveled}.} 
My meta advice is to be able to maintain discipline, but don't stop there!

\subsubsection{More Rules}

\begin{minipage}[t]{0.45\textwidth}

\begin{enumerate}
    \item You don’t have to dream
    \item Don’t seek happiness
    \item Remember, it’s all Luck
    \item Exercise
    \item Be Hard on Your Opinions
    \item Be a teacher
    \item Define yourself by what you love
    \item Respect People With Less Power Than You
    \item Don’t Rush
\end{enumerate}
Tim Minchin.

See his \href{https://www.youtube.com/watch?v=yoEezZD71sc}{speech} and/or read a \href{https://medium.com/speak-louder/9-lessons-on-life-from-tim-minchin-822ab14d92df
}{blogpost} about the details. 

\end{minipage}
\hfill
\begin{minipage}[t]{0.45\textwidth}

\begin{enumerate}
    \item We are imperfect, 
    \item (True) Friendship, 
    \item Know your Insanity, 
    \item Accept your idiocy, 
    \item Good Enough, 
    \item Beyond Romanticism, 
    \item Cheerful despair,  
    \item Transcend yourself.
\end{enumerate}

The Eight Rules of \href{https://en.wikipedia.org/wiki/The_School_of_Life}{The School of Life}.
(See in an \href{https://www.youtube.com/watch?v=1JCJVaK48RM}{animated form}.)

\end{minipage}


\begin{minipage}[t]{0.45\textwidth}

\begin{enumerate}
    \item Be adaptive
    \item Learn how to deal with failure
    \item Be a storyteller
    \item Get to know yourself
    \item Practice Vipassana meditation
    \item Engage with spirituality
    \item Study philosophy
    \item Read lots of books
    \item Develop your social skills
    \item Find your mission
    \item BONUS - Keep a broad perspective
\end{enumerate}

Collected \href{https://www.youtube.com/watch?v=2tEgpgeErRg}{here} from Yuval Noah Harari.
Talking to/with young students about \href{https://www.youtube.com/watch?v=j0uw7Xc0fLk}{The Future of Education} 

\end{minipage}
\hfill
\begin{minipage}[t]{0.45\textwidth}


\begin{enumerate}
    \item Happiness
    \item Achievement
    \item Significance
    \item Legacy
\end{enumerate}

A snippet from
\href{https://hbr.org/2004/02/success-that-lasts}{Success That Lasts}.
Further advice can be found in  the book:
\href{https://www.goodreads.com/book/show/13538833-howard-s-gift}{Uncommon Wisdom to Inspire Your Life's Work}, which is summarized in the presentation:
\href{https://www.youtube.com/watch?v=wLn28DrSF68}{Building a Life by Howard H. Stevenson}.
(I want to add, that if one has a dream, and a will to work on it, then it should not be forgotten, even if it is being and actor.)

\end{minipage}

\subsubsection{One ``self help'' book}

I put this here, because based on the \href{https://www.youtube.com/watch?v=lz8sUiXAnbs}{the authors summary} The Subtle Art of Not Giving a F*ck seems to be actual (in the current (2016) Western world for a middle-upper class reader), and it can embed some core concepts from long existing philosophical traditions into this context.
I don't recommend this for depth, but some concepts can be viewed as a fair first approximation for the complex art of living in our times.


\subsubsection{Commencement speeches}

\begin{itemize}
    \item \href{https://www.youtube.com/watch?v=UibfDUPJAEU}{J.K. Rowling}
    \item \href{https://www.youtube.com/watch?v=UF8uR6Z6KLc}{Steve Jobs} 
\end{itemize}




\subsubsection{Planning and writing about yourself}

\begin{itemize}
    \item \href{https://yearcompass.com/}{YearCompass}
    \item \href{http://web.archive.org/web/20210202023126/https://thecharacterarc.com/wp-content/uploads/2019/01/Future-Authoring-Planner-by-The-Character-Arc.pdf}{Future Authoring Planner}
\end{itemize}



\subsubsection{Motivation}

So I think motivational speeches and motivational speakers represent the fast food version of the mentioned concepts. My advise – being coherent with the opening quote – is to be suspicious with speakers, who claim that you can achieve (usually material) success without major failures, crises and hard work.
You will need to learn, and you will learn the most by failures. If you never fail, then you were not ambitious enough.
As you will learn and grow (or simply by being in a new environment) your value system will at least partially change. This will most probably induce an existential crisis. This can be painful, but it can cause a leap in your growth.
If somebody says, that you should not experience sadness, and gloominess, and you should change yourself to achieve material success, \href{https://www.youtube.com/watch?v=NA4uTFeHEFA}{run}.


\section{Replication manual}

\subsection{Used tools}

This document was made by \href{https://www.overleaf.com/}{Overleaf}, which uses \href{https://en.wikipedia.org/wiki/LaTeX}{LaTeX}.
One can find nice \href{https://www.overleaf.com/learn/latex/Tutorials}{tutorials} on Overleaf, to learn how to use it. The source of the project can be found on:
\begin{itemize}
    \item \href{https://www.overleaf.com/read/vjhvckyttxxx}{Overleaf}
    \item \href{https://github.com/konczer/OpenCurriculum}{GitHub}
    \item \href{https://web.archive.org/web/20211209191853/https://github.com/konczer/OpenCurriculum/blob/main/main.tex}{Wayback Machine}
\end{itemize}

\subsection{Licensing}

\href{https://creativecommons.org/publicdomain/zero/1.0/}{CC0 1.0 Universal (CC0 1.0)}

You are free to:
\begin{itemize}
    \item Share — copy and redistribute the material in any medium or format
    \item Adapt — remix, transform, and build upon the material
    for any purpose, even commercially.
\end{itemize}

See the full license \href{https://creativecommons.org/publicdomain/zero/1.0/legalcode}{here}.

\subsection{Instruction}

If you write your own document, consider to include a \bf{Replication manual}.

\newpage

\tableofcontents \label{sec:toc}

\end{document}
